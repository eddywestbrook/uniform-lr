
%% define an if for whether we are compiling as a tech-report
\newif\iftechreport
\techreportfalse
%\techreporttrue

%%% define the document class / layout for the tech report version... %%%
\iftechreport
\setlength{\itemsep}{0pt}
\documentclass{article}

%% margins
\setlength{\oddsidemargin}{0in}
\setlength{\textwidth}{6.5in}

%% paragraph layout
% \setlength{\parindent}{0pt}
% \addtolength{\parskip}{\baselineskip}

%% set list spacing parameters
\usepackage{tweaklist}
\renewcommand{\descripthook}{\setlength{\topsep}{10pt}%
  \setlength{\itemsep}{0pt}\setlength{\labelsep}{0pt}}

%%% and the non-tech-report version %%%
\else
\documentclass[conference,compsoc]{IEEEtran}
%\documentclass[a4paper,USenglish]{lipics}
\fi


\usepackage{amsmath}
\usepackage{amssymb}
\usepackage{stmaryrd}
\usepackage{proof}

\begin{document}

%%% my commands %%%


%%%
%%% spacing
%%%

\newlength{\indlen}
\setlength{\indlen}{8pt}
\newcommand{\ind}[1]{\hspace{#1\indlen}} % stands for indent
\newcommand{\figfill}[1][9pt]{\hspace{#1}\hfill}


%%%
%%% theorem/proof constructs
%%%

\newtheorem{theorem}{Theorem}
\newtheorem{lemma}{Lemma}
\newtheorem{definition}{Definition}
\newtheorem{corollary}{Corollary}

% proof environment
\iftechreport
\newenvironment{myproof}
               {\hspace{-1\parindent}\textbf{Proof:}}
               {\hfill $\square$}
\else
\newenvironment{myproof}
               {\begin{IEEEproof}}
               {\end{IEEEproof}}
\fi


%%%
%%% for writing grammars
%%%

\newcommand{\borspacing}{\hspace{.5em}}
\newcommand{\doublebar}{\protect\rule[-2pt]{1pt}{1em}\hspace{1pt}\protect\rule[-2pt]{1pt}{1em}}
\newcommand{\singlebar}{\protect\rule[-2pt]{1pt}{1em}}
\newcommand{\borstar}{\borspacing\singlebar}
%\newcommand{\borstar}{\doublebar\borspacing}
%\newcommand{\borstar}{|\borspacing}
\newcommand{\bor}{\borstar\borspacing} % stands for bar-or, for use in grammars


%%%
%%% for repetitive commands that just set the font face
%%%

\newcommand{\definermcommand}[2]{\newcommand{#1}{\ensuremath{\mathrm{#2}}}}
\newcommand{\definesfcommand}[2]{\newcommand{#1}{\ensuremath{\mathsf{#2}}}}
\newcommand{\definebfcommand}[2]{\newcommand{#1}{\ensuremath{\mathbf{#2}}}}


%%%
%%% type theory constructs
%%%

\definesfcommand{\Type}{Type}
\definesfcommand{\Prop}{Prop}
\definesfcommand{\Name}{Name}

\newcommand{\To}{\Rightarrow}
\newcommand{\lamabs}[3]{\lambda\,#1 \! : \! #2 \,.\, #3}
\newcommand{\lamabsnot}[2]{\lambda #1.\,#2}
\newcommand{\tlamabsnot}[2]{\Lambda #1.\,#2}
\newcommand{\lamgamma}[2][\Gamma]{\lambda #1\,.\,#2}
\newcommand{\nuabs}[3]{\nu\,#1 \! : \! #2 \,.\, #3}
\newcommand{\nugamma}[2]{\nu\,#1 \,.\, #2}
\newcommand{\nuabsnot}[2]{\nu\,#1\,.\,#2}
\newcommand{\pitype}[3]{\Pi #1\!:\!#2\,.\,#3}
\newcommand{\pitypenot}[2]{\Pi #1\,.\,#2}
\newcommand{\pigamma}[2]{\Pi #1\,.\,#2}
\newcommand{\nabtype}[3]{\nabla #1\!:\!#2\,.\,#3}
\newcommand{\nabgamma}[2]{\nabla #1\,.\,#2}
\newcommand{\namearg}[1]{\ensuremath{@\;#1}}
\newcommand{\namerepl}[2]{\ensuremath{#1\;\namearg{#2}}}
\newcommand{\subtype}{\ensuremath{<:}}
\newcommand{\ctorapp}[3]{#1(#2; #3)}
\newcommand{\uapp}[2]{#1(#2)}

%% sorts
\definermcommand{\impred}{i}
\definermcommand{\pred}{p}
\newcommand{\spred}{s^{\pred}}
\newcommand{\simpred}{s^{\impred}}
\newcommand{\sip}{s^{ip}}

%% pattern-matching functions
\newcommand{\funmatch}[5]{\mathbf{fun}\;#1\;(#2)\;(#3)\;(#4;#5)}
\newcommand{\funmatchi}[6]{\mathbf{fun}^{#1}\;#2\;(#3)\;(#4)\;(#5;#6)}
\newcommand{\pattern}[2]{#1 \backslash #2}
\newcommand{\pc}[3]{\pcnoctxt{\pattern{#1}{#2}}{#3}}
\newcommand{\pcnoctxt}[2]{#1 \to #2}
\newcommand{\pmfun}[4][]{\mathbf{fun}^{#1}\;#2\;(#3)\;(#4)}
\newcommand{\ppmfun}[3]{\pmfun[\pred]{#1}{#2}{#3}}
\newcommand{\ipmfun}[3]{\pmfun[\impred]{#1}{#2}{#3}}
\newcommand{\pmfunsimp}[1]{\mathbf{fun}\;(#1)}
\newcommand{\caseof}[2]{\mathbf{case}\;#1\;\mathbf{of}\;#2}

%% contexts
\newcommand{\gxform}[2][x]{#1:#2}
\newcommand{\guform}[4][u]{#1:^{#2;#3} #4}
\newcommand{\Gammaarg}{\Gamma_{\mathsf{arg}}}
\newcommand{\Gammap}{\Gamma_{\mathsf{p}}}
\newcommand{\Gammaa}{\Gamma_{\mathsf{a}}}
\newcommand{\Gammac}{\Gamma_{\mathsf{c}}}
\newcommand{\Gammaci}[1][i]{\Gamma_{\mathsf{c}_{#1}}}
\newcommand{\ctxtsel}[2]{#1|_{#2}}


%%%
%%% examples
%%%

\definesfcommand{\nat}{nat}
\definesfcommand{\z}{z}
\definesfcommand{\s}{s}
\definesfcommand{\False}{False}

\definesfcommand{\ord}{ord}
\definesfcommand{\ordz}{zero}
\definesfcommand{\ords}{succ}
\definesfcommand{\ordlim}{lim}


%%%
%%% static and operational semantics
%%%

\newcommand{\ctortp}[2]{(#1;\; #2)}
\newcommand{\typej}[3][\Gamma]{#1 \vdash #2 : #3}
\newcommand{\validj}[2][\Gamma]{#1 \vdash #2}
\newcommand{\rrto}{\longrightarrow}
\newcommand{\rrtostar}{\longrightarrow^{*}}
\newcommand{\conv}{\longleftrightarrow^{*}}
\newcommand{\convone}{\longleftrightarrow}
\newcommand{\uform}[4]{#1:^{#2;#3} #4}

%% special forms of reduction
\newcommand{\rrtocbn}{\rrto_{\mathrm{n}}}
\newcommand{\rrtocbnnc}{\rrto_{\mathrm{n-nc}}}

%% typing for contexts
\newcommand{\wfctxtj}[1]{\vdash #1}
\newcommand{\wfj}[2][\Gamma]{#1\vdash #2}


%%%
%%% helpers for interpretation function
%%%

%% ordered tuples
\newcommand{\onetuple}[1]{\langle #1\rangle}
\newcommand{\pair}[2]{\langle #1 ,\; #2\rangle}
\newcommand{\triple}[3]{\langle #1 ,\; #2, \; #3\rangle}

%% my own special logical operators, with extra spacing and text where needed
\newcommand{\mywedge}{\,\wedge\,}
\newcommand{\myvee}{\,\vee\,}
\newcommand{\myimplies}{\text{ \textbf{implies} }}

%% set comprehension
\newcommand{\setcompr}[2]{\{\, #1 \;|\;\; #2 \,\}}
\newcommand{\setcomprarr}[2]{\{\, #1 \;|\;\; \begin{array}[t]{@{}l@{}} #2 \,\}\end{array}}
\newcommand{\setcomprlong}[2]{\begin{array}[t]{@{}l@{}}\{\, #1 \\\;\;\;\;|\;\; \begin{array}[t]{@{}l@{}}#2 \,\}\end{array}\end{array}}

%% if-then-else
\newcommand{\ifthen}[3]{\text{\textbf{if} }#1\text{ \textbf{then} }#2\text{ \textbf{else} }#3}
\newcommand{\ifthennl}[3]{\text{\textbf{if} }#1\begin{array}[t]{@{}l@{}}\text{ \textbf{then} }#2\\\text{ \textbf{else} }#3\end{array}}

%% primitive recursion
\newcommand{\primrecname}[1][\dummy]{\mathbf{primrec}^{#1}}
\newcommand{\primrec}[4][\dummy]{\primrecname[#1]_{#2}\;(#3)\;(#4)}
\newcommand{\primrecarr}[4][\dummy]{\primrecname[#1]_{#2}\;(#3)\;(\begin{array}[t]{@{}l@{}}#4)\end{array}}
\newcommand{\primrecnl}[4][\dummy]{\primrecname[#1]_{#2}\;\begin{array}[t]{@{}l@{}}(#3)\\(#4)\end{array}}
% \newcommand{\primrecname}{\mathbf{primrec}}
% \newcommand{\primrec}[3]{\primrecname_{#1}\;(#2)\;(#3)}

%% "dummy" definitions
\newcommand{\lamstar}[3]{\lambda^{\dummy} (#1\in #2) .\; #3}
\newcommand{\lamstarnot}[2]{\lambda^{\dummy} #1 .\; #2}
\newcommand{\pistar}[2]{#1 \to^{\dummy} #2}
\newcommand{\appstar}[2]{#1 \;@^{\dummy}\; #2}
\newcommand{\ctorstar}[2]{#1^{\dummy}(#2)}
%\newcommand{\instar}{\in^{\dummy}}
\newcommand{\Deltadi}[1][i]{\Delta_{\dummy,#1}}
\newcommand{\sigmadi}[1][i]{\sigma_{\dummy,#1}}

%% forming Deltas and sigmas
\newcommand{\dxform}[3][x]{#1\mapsto (#2,#3)}
%\newcommand{\duform}[4][u]{#1:(#2;#3)\mapsto #4}
\newcommand{\duform}[3][u]{#1\mapsto (#2,#3)}
\newcommand{\dgform}[3][\Gamma]{#1\mapsto (#2,#3)}
\newcommand{\dgargform}[3]{#1\mapsto (#2,#3)}

%% well-formedness
%\newcommand{\wfdsj}[3][\Gamma]{#1 \vdash (#2;#3)}
%\newcommand{\wfdsjdef}{\wfdsj[\Gamma]{\Delta}{\sigma}}
\newcommand{\tctypej}[4][\Gamma]{#1 \vdash (#2 ; #3) : #4}
\newcommand{\tctypejdef}[1][\Gamma']{\tctypej{\Delta}{\sigma}{#1}}
\newcommand{\typejd}[2]{\typej[\dctxt{\Delta}]{#1}{#2}}
\newcommand{\typejemp}[2]{\typej[\cdot]{#1}{#2}}
\newcommand{\typejdg}[2]{\typej[\dctxt{\Delta},\Gamma]{#1}{#2}}
\newcommand{\validjd}[1]{\validj[\dctxt{\Delta}]{#1}}
\newcommand{\ivtypej}[3][\Gamma]{#1 \vdash #2 : #3}

%% misc
\newcommand{\Iarg}{I_{\mathsf{arg}}}
\newcommand{\size}[1]{\mathbf{size}(#1)}
\newcommand{\Mfun}{\ensuremath{M_{\mathrm{fun}}}}
\newcommand{\Afun}{\ensuremath{A_{\mathrm{fun}}}}
\newcommand{\proji}[1]{\pi_3 (#1)}
\newcommand{\proj}[2][c]{\mathsf{proj}_{#1,#2}}
\newcommand{\projvec}[1][c]{\vec{\mathsf{proj}_{#1}}}
\newcommand{\F}{\ensuremath{\mathfrak{F}}}
\newcommand{\Fomega}{\ensuremath{\F^{\omega}}}
\newcommand{\Ii}[1][i]{\ensuremath{\mathbb{I}_{#1}}}
\newcommand{\Sii}[2][i]{\ensuremath{\mathbb{S}_{#1,#2}}}
\newcommand{\Ui}[1][i]{\ensuremath{\mathbb{U}_{#1}}}
%\newcommand{\Ii}[1][i]{\ensuremath{\mathbb{I}_{#1}}}
%\newcommand{\Iomega}{\Ii[\omega]}
\newcommand{\liftset}[2]{\ensuremath{\mathcal{U}}_{#1}(#2)}
\newcommand{\powerset}{\ensuremath{\mathcal{P}}}
\definesfcommand{\Ctxt}{Ctxt}
\definesfcommand{\Term}{Term}
\newcommand{\interphole}[2][I]{\ensuremath{#1[#2]}}
\newcommand{\ctors}[1]{\mathrm{ctors}_{\Sigma}(#1)}
\newcommand{\elimok}[2]{\mathbf{elim\text{-}ok}(#1\mapsto #2)}
\definermcommand{\Dom}{Dom}
\definermcommand{\Ran}{Ran}
\newcommand{\countablefun}[2]{#1 \stackrel{\aleph_0}{\to} #2}
\definermcommand{\TC}{TC}
\newcommand{\instar}{\in^{*}}
\definesfcommand{\id}{id}

%% denotations
\newcommand{\denotationnot}[1]{\lceil #1 \rceil}
\newcommand{\denotation}[3]{\lceil #3 \rceil_{#1\vdash #2}}
\newcommand{\denotationd}[1]{\denotation[\dctxt{\Delta}]{#1}}
%\newcommand{\interpd}[2][\Delta]{\denotationd[\dctxt{#1},#2]{\interp[#1]{#2}}}
\newcommand{\interpd}[2][\iargs{\Delta}{\Xi}]{\llbrace #2 \rrbrace^{#1}}
%\newcommand{\interpdS}[2][\Delta]{\llbrace #2 \rrbrace^{#1}_{S}}
%\newcommand{\interpdI}[2][\Delta]{\llbrace #2 \rrbrace^{#1}_{I}}
\newcommand{\llbrace}{\{\hspace{-2.5pt}|}
\newcommand{\rrbrace}{|\hspace{-2.5pt}\}}

%%%
%%% interpretation and related constructs
%%%

%\newcommand{\interp}[2][\Delta]{\llbracket #2 \rrbracket^{#1}}
\newcommand{\iargs}[2]{#1;#2}
\newcommand{\interp}[2][\iargs{\Delta}{\Xi}]{\llbracket #2 \rrbracket^{#1}}
\newcommand{\interpnod}[1]{\interp[]{#1}}
\newcommand{\interpind}[1]{\llbracket #1 \rrbracket_{\mathbf{ind}}}
%\newcommand{\interpSarg}[3]{\llbracket #3 \rrbracket^{#2}_{#1}}
%\newcommand{\interpS}[2][\Delta]{\interpSarg{S}{#1}{#2}}
%\newcommand{\interpI}[2][\Delta]{\interpSarg{I}{#1}{#2}}
\newcommand{\dummy}{\ensuremath{\bullet}}
\newcommand{\dctxt}[1]{|#1|}
\newcommand{\Dsubst}[1][\Xi]{#1^{\sigma}}
\newcommand{\Dlookup}[2][\Xi]{#1^{\mathrm{I}}(#2)}
\definesfcommand{\SN}{SN}
\definesfcommand{\NF}{NF}
\newcommand{\CRnot}[1]{\ensuremath{\mathbf{CR}(#1)}}
\newcommand{\CR}[3]{\ensuremath{\mathbf{CR}_{#1\vdash#2}(#3)}}
\newcommand{\CRone}{\textbf{CR~1}}
\newcommand{\CRtwo}{\textbf{CR~2}}
\newcommand{\CRthree}{\textbf{CR~3}}
\newcommand{\CRfour}{\textbf{CR~4}}
\newcommand{\CRG}{\textbf{G}}
\newcommand{\CRT}{\textbf{T}}
\newcommand{\CRE}{\textbf{E}}


%%%
%%% Conclusion stuff about excluded middle
%%%

\newcommand{\ZFi}[1][i]{\ensuremath{\text{ZF}_{#1}}}
\newcommand{\ZFCi}[1][i]{\ensuremath{\text{ZFC}_{#1}}}

%%% Local Variables: 
%%% mode: latex
%%% TeX-master: "uniform-lr"
%%% End: 



\title{Uniform Logical Relations}

\iftechreport
\author{Edwin Westbrook\\
Department of Computer Science\\
Rice University\\
Houston, TX 77005\\
Email: \texttt{emw4@rice.edu}}

\else
% \author[1]{Edwin Westbrook}
% \affil[1]{Department of Computer Science\\
% Rice University\\
% Houston, TX 77005\\
% Email: \texttt{emw4@rice.edu}}
\author{\IEEEauthorblockN{Edwin Westbrook}
\IEEEauthorblockA{Department of Computer Science\\
Rice University\\
Houston, TX 77005\\
Email: emw4@rice.edu}}
\fi

\maketitle

\begin{abstract}
  Strong Normalization (SN) is an important property for intensional
  constructive type theories such as the Calculus of Inductive
  Constructions (CiC), the basis for the Coq theorem prover. Not only
  does SN imply consistency, but it also ensures that type-checking is
  decidable, and further, it provides a straightforward model, the
  term model, for a theory. Unfortunately, although SN has been proved
  for fragments of CiC, it is not known how to prove SN for CiC in its
  entirety, including eliminations for large inductive types as well
  as higher predicative universes. In this work, we show how to prove
  SN for full CiC. They key insight given here is that terms must be
  interpreted in a \emph{uniform} manner, meaning that the form of the
  interpretation of a term must not depend on whether the term is a
  type. We introduce a new technique called Uniform Logical Relations,
  With uniformity as a guiding principle, and we show that this
  technique can then be used to prove SN for CiC. An important
  property of our technique is that it does not rely on Confluence,
  and thus it could potentially be used for extensions of CiC with
  added computation rules, such as Extensionality, for which
  Confluence relies on SN.
\end{abstract}

%\IEEEpeerreviewmaketitle


%%%%%%%%%%%%%%%%%%%
%%%%% Section %%%%%
%%%%%%%%%%%%%%%%%%%
\section{Introduction}

Intensional constructive type theory (ICTT), as first introduced by
Martin-L\"{o}f \cite{martinlof72}, is compelling because it gives a
combined programming language and mathematical theory. This allows
users to write programs and proofs about those programs in the same
language (see e.g.~\cite{nps90}). Further, theories based on ICTT are
defined entirely syntactically. This makes them foundational, in the
sense that they can be defined with little or no appeal to models
or set theory.

An important consideration for such theories is Strong Normalization
(SN), which states that no well-typed term of the theory has an
infinite reduction path. Stated differently, SN says that the programs
of a theory have no infinite loops. SN is useful for a number of
reasons. First, it ensures the consistency of a theory, as it implies
that every proof has a cut-free version. Second, it ensures that
type-checking is decidable. Finally, SN also gives a simple model of
the theory, namely, the normal forms of that theory.

Unfortunately, it is not known how to prove SN for more powerful
extensions of ICTT such as the Calculus of Inductive Constructions
(CiC), the basis of the Coq theorem prover \cite{coq}. Although there
have been numerous proofs of SN for fragments of CiC
\cite{geuvers94,goguen94,altenkirch93,geuvers91,luo90,coquand90}, none
of these approaches has been generalized to the entirety of CiC. This
has lead to research into other approaches to proving the consistency
of CiC by providing set-theoretic models
\cite{barras10,werner08,miquel02,miquel00,werner97,stefanova96}.
The key difficulty is in combining two features: an impredicative
universe $\Prop$, and a cumulative type hierarchy of universes
$\Type_i$ for $i\geq 0$.  To handle impredicativity requires the
Logical Relations technique introduced by Girard for System F
\cite{girard-proofs-types}, in which each type is interpreted as a
logically defined set of terms, also called a logical relation.  With
a cumulative type hierarchy, however, we do not know whether a type is
a sort or not unless we normalize it, and thus, under the standard
approach to Logical Relations, it is unknown how to even interpret a
type without normalizing it. This seems to require SN to prove SN! The
problem is especially difficult in the presence of inductive types (as
pointed out by e.g.~\cite{geuvers94}), since the form of a type can
depend on whether a natural number term evaluates to 0 or to
1. Although Luo showed how to break this cycle in the case of ECC
using a difficult and complex quasi-normalization theorem
\cite{luo90}, ECC contains only products, not inductive types, and it
is not clear that this approach generalizes to theories that do
contain inductive types.


In this paper, we show how to solve this problem by using an
interpretation that is \emph{uniform}, meaning that the interpretation
does not need to know whether a type is a sort.\footnote{This approach
  came out of the author's dissertation, which gave an interpretation
  of a nominal extension of CiC into CiC \cite{westbrook-thesis}; the
  interpretation for CiC in set theory given here is actually much
  simpler because we do not have to worry things like intensional
  equality and type universes in set theory.}  We call this approach
Uniform Logical Relations.  The key technical development that ensures
uniformity is to interpret types as sets of triples of context, terms, and
interpretations, i.e., sets of the form
$\setcompr{\triple{\Gamma}{M}{I}}{\phi(\Gamma,M,I)}$ where $M$ is a term
that is well-typed with respect to $\Gamma$,
and $\phi$ is
a logical formula which states that $I$ is a valid interpretation of
$M$. This is as opposed to the usual approach of interpreting types as
sets of terms $\setcompr{M}{\phi(M)}$.
% (Technically, both our
% approach and many prior approach consider not just terms but ``terms
% in context''; i.e., each term in a set is also bundled with a context
% $\Gamma$ relative to which it is well-typed.)
Under
the usual approach, the interpretation of the function type
$\pitype{x}{A}{B}$ depends on the form of $A$: this interpretation
must quantify over terms of type $A$ and
reducibility candidates when $A$ is a sort; over terms of type $A$ and
functions on reducibility candidates when $A$ is a kind-level
$\Pi$-type; and over just terms of type $A$ when $A$ is a type.  This
information about $A$ requires the normal form of $A$ in general.
Under Uniform Logical Relations, in contrast, the interpretation of
$\pitype{x}{A}{B}$ always quantifies over triples
$\triple{\Gamma}{N}{X}\in\interpnod{A}$. This uniformity means we do not require
the normal form of $A$.


From a different viewpoint, uniformity means the interpretation
captures ``deep'' information about every term in the theory, instead
of just the types. For example, the interpretation of a constructor
application such as $c\;M_1\;M_2$ is the tuple $\triple{c}{I_1}{I_2}$,
where each $I_i$ is the interpretation of $M_i$. Intuitively, this
deep information is needed because even ``simple'' objects, like
natural numbers, can influence the interpretation of a type built
using elimination forms, such as the type
\[
\caseof{x}{\pcnoctxt{\z}{\nat} \;|\; \pcnoctxt{\s\;y}{\nat\to\nat}}
\]

A central concern with deep interpretations inside ZF set theory is
the Axiom of Regularity, which states that the set membership relation
$\in$ is well-founded. On the positive side, Regularity immediately
justifies primitive recursion on inductive types, because recursive
calls in CiC are guaranteed to follow the transitive closure of the
$\in$ relation.  More subtly, Regularity explains why Uniform Logical
Relations does not work for potentially non-terminating terms like
$\lamabsnot{x}{x\;x}$, where a term is applied to itself, since no
function can be in its own domain or range without creating a cycle in
the $\in$ relation.


On the negative side, however, Regularity makes it difficult to
interpret proofs because of impredicativity, as proofs can quantify
over domains that contain themselves. To break the ``vicious cycle''
of impredicativity, all proofs are given a ``dummy'' interpretation,
written $\dummy$, which intuitively contains no information.  (The
empty set $\emptyset$ could be used here, but we use a separate
notation $\dummy$ for clarity.)  This works because it is not possible
in CiC to perform eliminations of proofs except when either the input
type intuitively contains no information, like the empty type
$\False$, or the output is another proof, whose interpretation
is also a dummy and does not need any information from the input.
This technique also means that Uniform Logical Relations construction
satisfies the Proof Irrelevance property (see, e.g., \cite{miquel02}),
since all proofs have the same interpretation.


The remainder of this document is organized as follows. Section
\ref{sec:cic} reviews the syntax and semantics of the Calculus of
Inductive Constructions (CiC).  Section \ref{sec:cic-sn} then proves
Strong Normalization for CiC using Uniform Logical Relations.
% \footnote{Although it is widely believed that CiC has
%   equivalent proof-theoretic strength as ZF with inaccessible
%   cardinals, and thus that a proof of SN inside ZF would violate
%   G\"{o}del's Second Incompleteness theorem, this is in fact only true
%   for \emph{constructive} ZF; we discuss this more in Section
%   \ref{sec:conclusion}.}
% Section \ref{sec:cnic} describes the Calculus of Nominal
% Inductive Constructions (CNIC), an extension of CiC with support for
% nominal features and higher-order encodings.  Section
% \ref{sec:cnic-sn} then shows how Uniform Logical Relations can be
% adapted for CNIC.
Finally, Section \ref{sec:related-work} relates Uniform Logical
Relations to previous approaches, and Section \ref{sec:conclusion}
concludes.
% FIXME: cite tech report...?
% \iftechreport\else
% Proofs of most theorems are omitted for space reasons, but can be
% found in the companion technical report \cite{westbrook-techreport}.
% \fi



%%%%%%%%%%%%%%%%%%%
%%%%% Section %%%%%
%%%%%%%%%%%%%%%%%%%
\section{The Calculus of Inductive Constructions}
\label{sec:cic}

In this section, we give a brief overview of the Calculus of Inductive
Constructions. The version we give here has some small modifications
relative to the theory as presented in, for example, the Coq manual
\cite{coq}. These modifications are for increased conciseness, but do
not change the theory in a material way.

\textbf{A Note on Syntax:} Here and in the below we use particular
letters to stand for elements of particular syntactic classes; e.g.,
$M$ is always used for a term. These can appear with subscripts or
primes, as in $M_2'$ or $\Mfun$. We also use an arrow over a syntactic
construct to indicate a sequence of zero or more of those constructs.
For example, $\vec{x}$ denotes a sequence of variables. We use
subscripts $i$ and $j$ to denote the $i$th and $j$th elements of this
sequence, e.g., $x_i$, and we use vertical bars $|\vec{x}|$ to denote
the length of this sequence. If the letter under the arrow has a
prime, then so do the elements of the sequence, so for instance the
elements of $\vec{M'}$ are referred to as $M'_i$.  We also use the
following compound notations: $M\;\vec{M}$ denotes the multi-arity
application $M\;M_1\ldots M_{|\vec{M}|}$; $[\vec{M}/\Gamma]$ denotes
the multi-arity substitution
$[M_1/x_1,\ldots,M_{|\vec{M}|}/x_{|\vec{M}|}]$, where $\vec{x}$ are
the variables on the left in $\Gamma$; and
$\pc{\vec{c}}{\vec{\Gamma}}{\vec{M}}$ denotes the sequence of pattern
cases $\pc{c_1}{\Gamma_1}{M_1}\;|\ldots|\;\pc{c_1}{\Gamma_1}{M_1}$

We assume mutually disjoint sets of constructors $c$, type
constructors $a$, variables $x$, and recursive variables $u$, where
the latter are used for recursive calls in primitive recursive
pattern-matching functions and the sets of variables and recursive
variables are infinite. We also use $y$ and $z$ for variables.

Figure \ref{fig:syntax} gives the syntax of CiC. The signatures
$\Sigma$ associate types with constructors $c$ and type constructors
$a$, while the contexts $\Gamma$ map variables $x$ and recursive
variables $u$ to types. We also use $\Delta$ for contexts below.
Recursive variables $u$ are also associated
with a number $n$ and a sequence of variables $\vec{x}$, which specify
that all occurrences of $u$ should have the form $u\;\vec{M}\;x_i$ for
some $i$ where $|\vec{M}|=n$, to ensure termination of primitive
recursion. In the below we write $|\Gamma|$ for the length of
$\Gamma$, i.e., the number of commas.


%%% syntax figure %%%
\begin{figure}
\centering
\iftechreport\else\begin{small}\fi
\begin{math}
\begin{array}{@{}lrl@{}}
\Sigma & ::= & \cdot \bor \Sigma,c:M \bor \Sigma,a:M\\
\Gamma & ::= & \cdot \bor \Gamma,\gxform{M} \bor \Gamma,\guform{n}{\vec{x}}{M}\\
M & ::= & \Prop \bor \Type_i \bor \pitype{x}{M}{M}
\bor \ctorapp{a}{\vec{M}}{\vec{M}}\\
& \borstar & x \bor M\;M \bor \lamabs{x}{M}{M} \bor \ctorapp{c}{\vec{M}}{\vec{M}}\\
& \borstar & u \bor \pmfun[ip]{u}{\Gamma}{\pc{c}{\Gamma}{M}\;|\;\ldots\;|\;\pc{c}{\Gamma}{M}}
\end{array}
\end{math}
\iftechreport\else\end{small}\fi
\caption{Syntax of CiC}
\label{fig:syntax}
\end{figure}


The remainder of Figure \ref{fig:syntax} defines the terms $M$.  We
also use $N$, $Q$, and $R$ for terms, as well as $A$ and $B$ for terms
that are meant to denote types.  The terms include the impredicative
universe $\Prop$ and the predicative universes $\Type_i$ for all
$i\geq 0$. These are the \emph{sorts}, written $s$.  We also sometimes
use $ip$ for meta-variable that varies over $\{\pred,\impred\}$.  In
addition, $\Type_{-1}$ is also used to denote $\Prop$.  Terms also
include the function types $\pitype{x}{A}{B}$ and the inductive types
$\ctorapp{a}{\vec{M}}{\vec{N}}$, where the latter distinguishes
between the \emph{parameters} $\vec{M}$ and the \emph{arguments}
$\vec{N}$. Next are the variables $x$, the applications $M\;N$, the
$\lambda$-abstractions $\lamabs{x}{A}{M}$, and the constructor
applications $\ctorapp{c}{\vec{M}}{\vec{N}}$, which again
distinguish between parameters and arguments.

The final line lists the recursive variables $u$ and the
pattern-matching functions, where the latter have the form
$\pmfun[ip]{u}{\Gammaarg}{\pc{\vec{c}}{\vec{\Gamma}}{\vec{M}}}$.
Pattern-matching functions combine the \texttt{Fix} and \texttt{case}
terms of Coq. Each compound form $\pc{c_i}{\Gamma_i}{M_i}$ is called a
\emph{pattern case} with \emph{pattern context} $\Gamma_i$ and
\emph{body} $M_i$.  Pattern-matching functions of this form take in
$|\Gammaarg|+1$ arguments and pattern-match on the last argument. The
last argument is called the \emph{scrutinee} and the earlier arguments
are called the \emph{parameters}. If the scrutinee has the form
$\ctorapp{c_i}{\vec{N}}{\vec{Q}}$ then the pattern-matching function
returns $M_i$, substituting the parameters for the variables of
$\Gammaarg$ and substituting the arguments $\vec{Q}$ for the pattern
variables listed in $\Gamma_i$. Pattern-matching functions are also
annotated with either $\pred$ or $\impred$, indicating whether they
eliminate a predicative or impredicative inductive type.


The operational semantics of CiC is given as a higher-order rewrite
system $\rrto$ in Figure \ref{fig:opsem}. This means that $M_1\rrto
M_2$ iff $M_2$ can be obtained by replacing a subterm of $M_1$ that
matches the left-hand side of a rule listed in Figure \ref{fig:opsem}
be the corresponding right-hand side of the rule.  The first rule
performs the usual $\beta$-reduction, while the second performs the
pattern-matching reduction described in the previous paragraph.  In
the below, we write $\rrtostar$ and $\conv$ respectively
for the reflexive-transitive closure
and the reflexive-symmetric-transitive closure of $\rrto$.


%%% Operational Semantics %%%
\begin{figure}
\centering
\iftechreport\else\begin{small}\fi
\begin{math}
\begin{array}{@{}rcl@{}}

\hspace{20pt}(\lamabs{x}{A}{M})\;N & \rrto & [N/x]M
\\[5pt]

\multicolumn{3}{l}{
  (\Mfun=\pmfun{u}{\Gammaarg}{\pc{\vec{c}}{\vec{\Gamma}}{\vec{M}}})\;
           \vec{N}\;\ctorapp{c_i}{\vec{Q}}{\vec{R}}
%\hspace{47pt}
} \\
& \rrto & 
  [\vec{N}/\Gammaarg,\vec{R}/\Gamma_i,\Mfun/u]M_i
\end{array}
\end{math}
\iftechreport\else\end{small}\fi
\caption{Operational Semantics of CiC}
\label{fig:opsem}
\end{figure}


% \begin{lemma}[Confluence]
%   \label{lemma:confluence}
%   If $M\conv N$ then there exists $Q$ such that $M\rrtostar Q$ and
%   $N\rrtostar Q$.
% \end{lemma}

% \begin{myproof}
%   The relation $\rrto$ is defined by an orthogonal, higher-order
%   rewrite system and so is confluent by known techniques (see, e.g.,
%   \cite{baader-nipkow98}).
% \end{myproof}


\iftechreport\else
In the below we also make use of call-by-name (CBN) reduction.
Intuitively, $M$ CBN reduces to $N$, written $M\rrtocbn N$, iff $N$
can be reached from $M$ by only reducing a redex that is either to the
left of zero or more applications in $M$ or is in the scrutinee of a
pattern-matching function at the top level of $M$.  When $M\rrtocbn
M'$ holds by contracting a redex $R$ in $M$, we call the immediate
subterms of $R$ the \emph{constituents} of the reduction, and we
further say that $M$ call-by-name reduces to $M'$ \emph{with
  normalizing constituents}, written $M\rrtocbnnc M'$, if all these
constituents are strongly normalizing. If there is no $M'$ such that
$M\rrtocbn M'$ then $M$ is called a \emph{weak head normal form}
(WHNF).  These notions will be useful in Section \ref{subsec:interp}
below.
\fi



We do not give the well-formedness rules for signatures $\Sigma$;
these may be found in the Coq manual \cite{coq}. At a high level,
however, these require that $\Sigma$ contain sequences that define a
type constructor $a$ and its constructors $c_i$ of the following form:
\[
a:(\pigamma{\Gammap}{\pigamma{\Gammaa}{s}}),
c_1:(\pigamma{\Gammap}{\pigamma{\Gammaci[1]}{\ctorapp{a}{\Gammap}{\vec{M_c}}}}),
\ldots
\]
The context $\Gammap$ lists the \emph{parameters} of the inductive
type $a$, which must be the same for for all of the constructors.  The
contexts $\Gammaa$ and $\Gammaci[i]$ list the \emph{arguments} of $a$
and $c_i$, respectively. All of the types must be well-typed for the
prefix of $\Sigma$ before $a$, except that the $\Gammaci[i]$ may
contain $a$ \emph{strictly positively}, meaning that types of the form
$\ctorapp{a}{\vec{N}}{\vec{Q}}$ can only occur as the return type
of zero or more $\Pi$-abstractions directly to the right
of a colon in $\Gammaci[i]$. The signature $\Sigma$ and its
well-formedness are left implicit in the below.
We also implicitly assume that any contexts $\Gamma$ are well-formed,
meaning that all listed types are well-typed.

% Well-formedness for contexts is given by the judgment $\wfctxtj{\Gamma}$,
% defined as follows:\\
% \[
% \figfill
% \infer{\wfctxtj{\cdot}}{}
% \figfill
% \infer{\wfctxtj{\Gamma,\gxform{A}}}{\wfctxtj{\Gamma} & \typej{A}{s}}
% \figfill
% \infer{\wfctxtj{\Gamma,\guform{n}{\vec{x}}{A}}}{\wfctxtj{\Gamma} & \typej{A}{s}}
% \figfill
% \hspace{30pt}
% \]
% These rules require that each type $A$ listed in $\Gamma$ is
% well-typed at some sort $s$ in the prefix of $\Gamma$ before $A$.  We
% implicitly assume well-formedness of all contexts used in the typing
% rules below.


The typing rules are given in Figure \ref{fig:typing}. The first rule
is the conversion rule, which states that equal types have the same
elements. Note that this only allows a single step of conversion;
although the rule may be used multiple times, each use requires the
destination type $A'$ to be well-typed, ensuring that only conversions
that use well-typed types can be used.  The second rule is the
subtyping rule, which states that if $M$ has type $A$ for $A$ a
\emph{subtype} of $A'$, written $A\subtype A'$, then $M$ has type
$A'$.  Subtyping is the reflexive-transitive closure of the following
three cases: $\Prop\subtype\Type_0$; $\Type_i\subtype\Type_{i+1}$; and
if $B_1\subtype B_2$ then
$\pitype{x}{A}{B_1}\subtype\pitype{x}{A}{B_2}$.  Note that subtyping
is guaranteed to preserve well-typedness of types.  The next two rules
state that $\Prop:\Type_0$ and $\Type_i:\Type_{i+1}$. The next two
rules type $\Pi$-types $\pitype{x}{A}{B}$ by requiring the types of
both $A$ and $B$ to be sorts. Also, if $B$ is impredicative, meaning
it has type $\Prop$, then the type of $\pitype{x}{A}{B}$ is $\Prop$,
and otherwise the type of $\pitype{x}{A}{B}$ is the maximum universe
of that of both $A$ and $B$.


%%% Typing %%%
\begin{figure*}
\centering
\iftechreport\else\begin{footnotesize}\fi
\begin{tabular}{@{\hspace{-0pt}}c@{\hspace{-0pt}}}

\infer{\typej{M}{A'}}{\typej{M}{A} & A\conv A' & \typej{A'}{s}}
\figfill

\infer{\typej{M}{A'}}{\typej{M}{A} & A\subtype A'}
\figfill

\infer{\typej{\Prop}{\Type_0}}{}
\figfill

\infer{\typej{\Type_i}{\Type_{i+1}}}{}
\\ \\

\infer{\typej{\pitype{x}{A}{B}}{\Type_i}}{\typej{A}{\Type_i} & \typej[\Gamma,x:A]{B}{\Type_i}}
\figfill

\infer{\typej{\pitype{x}{A}{B}}{\Prop}}{\typej{A}{s} & \typej[\Gamma,x:A]{B}{\Prop}}

\figfill
\infer{\typej{\lamabs{x}{A}{M}}{\pitype{x}{A}{B}}}{\typej{A}{s} & \typej[\Gamma,x:A]{M}{B}}
\figfill

\infer{\typej{x}{A}}{x:A\in\Gamma}
\\ \\

\infer{\typej{M\;N}{[N/x]B}}{\typej{M}{\pitype{x}{A}{B}} & \typej{N}{A}}
\figfill

\infer{\typej{\ctorapp{a}{\vec{M}}{\vec{N}}}{s}}{
  a:\pigamma{\Gammap}{\pigamma{\Gammaa}{s}}\in\Sigma
  & \typej{(\vec{M};\vec{N})}{\Gammap;\Gammaa}
}
\\ \\

\infer{
  \typej{\ctorapp{c}{\vec{M}}{\vec{N}}}{\ctorapp{a}{\vec{M}}{[\vec{M}/\Gammap,\vec{N}/\Gammac]\vec{Q}}}
}{
  c:\pigamma{\Gammap}{\pigamma{\Gammac}{\ctorapp{a}{\Gammap}{\vec{Q}}}} \in\Sigma
  &
  \typej{(\vec{M};\vec{N})}{\Gammap;\Gammac}
}
\figfill[4pt]

\infer{\typej{u\;\vec{M}\;(x_i\;\vec{N})}{[(\vec{M},(x_i\;\vec{N}))/\Gamma_u]B}}{
  (\guform{|\vec{M}|}{\vec{x}}{\pigamma{\Gamma_u}{B}})\in\Gamma &
  \typej{(\vec{M},(x_i\;\vec{N}))}{\Gamma_u}
}
\\ \\

\infer{
  \typej{
    \pmfun[ip]{u}{\Gammaarg}{\pc{\vec{c}}{\vec{\Gamma}}{\vec{M}}}
  }{
    (\Afun = \pigamma{\Gammaarg}{\pitype{x}{\ctorapp{a}{\vec{N}}{\vec{Q}}}{B}})
  }
}{
  \begin{array}{@{}c@{}}
    \typej{\Afun}{s}
    \figfill
    a:\pigamma{\Gammap}{\pigamma{\Gammaa}{s_a^{ip}}}\in\Sigma
    \figfill
    \elimok{a}{s}
    \figfill
    \vec{c} = \ctors{a}
    \figfill
    \forall i. (\wfj[\Gamma,\Gammaarg]{\Gamma_i})
    \\
    \forall i. (\typej[\Gamma,\Gammaarg,\Gamma_i]{\ctorapp{c_i}{\vec{N}}{\Gamma_i}}{\ctorapp{a}{\vec{N}}{\vec{Q}}})
    \figfill
    \forall i. (\typej[\Gamma,\Gammaarg,\Gamma_i,\guform{|\Gammaarg|}{\Gamma_i}{}]{M_i}{[\ctorapp{c_i}{\vec{N}}{\Gamma_i}/x]B})
  \end{array}
}
\end{tabular}
\iftechreport\else\end{footnotesize}\fi
\caption{Typing Rules of CiC}
\label{fig:typing}
\end{figure*}



For inductive types $\ctorapp{a}{\vec{M}}{\vec{N}}$, the parameters
$\vec{M}$ and the arguments $\vec{N}$ must have the types given by the
contexts $\Gammap$ and $\Gammaa$ in $\Sigma$.  This is stated with the
judgment $\typej{(\vec{M};\vec{N})}{\Gammap;\Gammaa}$ which states
that $|\vec{M}|=|\Gammap|$, $|\vec{N}|=|\Gammaa|$, and that, for each
term $Q$ in $\vec{M},\vec{N}$ with type $A$ occurring at the same
position in $\Gammap,\Gammaa$, we have
$\typej{Q}{[\vec{M}/\Gammap,\vec{N}/\Gammaa]A}$.

For variables, we look the type up in the context. For applications
$M\;N$, we require $M$ to have function type $\pitype{x}{A}{B}$ and
$N$ to have type $A$, returning type $[N/x]B$. For
$\lambda$-abstractions, we require the domain type $A$ to be
well-formed, the body $M$ to have type $B$ under the context extended
with variable $x$, and we return function type $\pitype{x}{A}{B}$.
For constructor applications $\ctorapp{c}{\vec{M}}{\vec{N}}$, we again
require the parameters $\vec{M}$ and arguments $\vec{N}$ to have the
appropriate types given in $\Gammap$ and $\Gammac$, yielding the
return type of $c$ with the parameters and arguments substituted in.
For recursive calls to $u$, we require the term to be of the form
$u\;\vec{M}\;x_i$ where $x_i$ is one of the variables to which $u$ may
be applied, where $|\vec{M}|$ is the number of arguments before $x_i$
to which $u$ must be applied, and where all of $\vec{M}$ and $x_i$
have the types required by the type of $u$.


To type a pattern-matching function, we first require that its type
$\Afun$ (which includes the parameter types $\Gammaarg$) has type $s$
for some $s$. We then require that $a$ can be eliminated at sort
$s$, written $\elimok{a}{s}$; this states that, if $a$ is
impredicative then $s=\Prop$, except for the special case that $a$ has
at most one constructor, all of whose arguments are proofs (i.e., have
a type which has type $\Prop$), when $s$ is allowed to be
predicative. Next, we require that the patterns exactly match
the constructors of $a$ in $\Sigma$, written
$\ctors{a}$. Next, we require that the pattern contexts $\Gamma_i$ are
all well-formed. Finally, we require that each $c_i$ applied to the
given parameters and the variables listed in $\Gamma_i$ has same type
as the input type, and that each body $M_i$ has as its type the result
of substituting this application of $c_i$ into the return type $B$.


\begin{lemma}
  \label{lemma:type-typing}
  If $\typej{M}{A}$ then $\exists s.\typej{A}{s}$.
\end{lemma}

\iftechreport

%%%%%%%%%%%%%%%%%%%%%%%
%%%%% Sub-Section %%%%%
%%%%%%%%%%%%%%%%%%%%%%%
\subsection{Call-by-Name Reduction}
\label{subsec:cbn}

In order to prove reducibility of $\lamabs{x}{A}{M}$ below, we will
need to prove that $(\lamabs{x}{A}{M})\;N$ is reducible for all
reducible $N$.  Since reducibility is proved by induction on the
structure of terms, we get by assumption that $[N/x]M$ is reducible;
to proceed, we need to ``undo'' a step of reduction.\footnote{This
  ``undoing'' is used in a similar manner as the condition \CRthree\
  in Girard's original proof \cite{girard-proofs-types}, and also has
  a counterpart in approaches based on saturated sets, such as that of
  Altenkirch \cite{altenkirch93}.} For pattern-matching functions,
however, the situation is more complex: if $\Mfun$ is a
pattern-matching function with cases $M_i$, and if we have by the
inductive hypothesis that $[\vec{N}/\Gammaarg,\vec{Q}/\Gamma_i]M_i$ is
reducible for all reducible $\vec{N}$ and $\vec{Q}$, then undoing a
step of reduction only yields that $\Mfun\;\vec{N}\;N$ is reducible
for $N$ of the form $\ctorapp{c}{\vec{R}}{\vec{Q}}$.  In other words,
we might need to undo multiple reductions on the scrutinee argument
$N$. We only need to undo reductions on the path to a weak head normal
form of $N$, however. Call-by-name (CBN) reduction gives a disciplined
way to talk about exactly these sorts of reductions.


In the remainder of this section, we introduce CBN reduction and
prove some properties of CBN that will be used below.  More
specifically, we deal with the reduction relation $\rrtocbnnc$ which
only performs CBN reductions when the immediate subterms of the redex
involved are all strongly normalizing; otherwise we could have
$M\rrtocbn N$ via a redex that ``throws away'' a subterm of $M$, so
$\SN(N)$ does not imply $\SN(M)$.  We then prove that undoing
$\rrtocbnnc$ preserves SN (Lemma \ref{lemma:cbn-nc-whnf}), and also
that $\rrtocbnnc$ satisfies a weak Standardization property for
reduction to weak head normal forms (Lemma \ref{lemma:cbn-nc-whnf}).
These both follow from Lemma \ref{lemma:cbn-diamond}, which shows that
$\rrtocbnnc$ weakly commutes with standard reduction. Note that none
of this material should be considered surprising or novel; Lemma
\ref{lemma:cbn-diamond}, for example, is proved as Lemma 3.5.2 by
Altenkirch \cite{altenkirch93}, where CBN reduction is called weak
head reduction.


%% CBN reduction
\begin{definition}[CBN Reduction]
  \label{def:cbn}
  We say that $M$ \emph{call-by-name (CBN) reduces} to $M'$, written
  $M\rrtocbn M'$, iff the following hold:
  \begin{enumerate}
  \item If $M$ is a redex, then $M'$ is the result of contracting $M$;
  \item If $M\equiv\Mfun\;\vec{N}\;N$ such that $M$ is not a redex and
    $\Mfun$ is a pattern-matching function with $|\vec{N}|$
    parameters, and if $N\rrtocbn N'$, then
    $M'\equiv\Mfun\;\vec{N}\;N'$; and
  \item If $M\equiv M_1\;M_2$ for $M$ not of one of the above forms,
    and if $M_1\rrtocbn M_1'$, then $M'\equiv M_1'\;M_2$.
  \end{enumerate}
  If none of these apply, then there is no $M'$ such that
  $M\rrtocbn M'$, and $M$ is called a \emph{weak head normal form}
  (WHNF). When $M\rrtocbn M'$ holds, by contracting a redex $R$ in
  $M$, we call the immediate subterms of $R$ the \emph{constituents}
  of the reduction, and we further say that $M$ call-by-name reduces
  to $M'$ \emph{with normalizing constituents}, written $M\rrtocbnnc
  M'$, if all these constituents are strongly normalizing.
\end{definition}

%% CBN-NC closed under application
\begin{lemma}
  \label{lemma:cbn-nc-app}
  If $M\rrtocbnnc M'$ then $M\;N\rrtocbnnc M'\;N$.
\end{lemma}

\iftechreport
\begin{myproof}
  If $M\rrtocbnnc M'$ then $M$ must be of the form $Q\;\vec{R}$ where
  $Q$ is either a redex or matches case 2 of Definition \ref{def:cbn}.
  In either case, $M\;N$ cannot be a redex, nor can it match case 2 of
  Definition \ref{def:cbn}, and so by case 3 of Definition
  \ref{def:cbn} we have that $M\;N\rrtocbn M'\;N$. In addition, this
  reduction has the same normalizing constituents as the original CBN
  reduction, so $M\;N\rrtocbnnc M'\;N$.
\end{myproof}
\fi

%% Diamond-y property of CBN
\begin{lemma}
  \label{lemma:cbn-diamond}
  If $M\rrtocbnnc N$ and $M\rrto M'$, then either $M'\equiv N$ or
  there exists $N'$ such that $M'\rrtocbnnc N'$ and $N\rrtostar N'$,
  where the latter can be strengthened to $N\rrto N'$ when the
  reduction $M\rrto M'$ contracts a redex of $M$ that is not a subterm
  of a constituent of the reduction $M\rrtocbnnc N$.
\end{lemma}

\iftechreport
\begin{myproof}
  Let $R$ be the redex contracted by the reduction $M\rrtocbnnc N$ and
  $R'$ be that contracted by the reduction $M\rrto M'$. We now
  consider four cases. If $R$ and $R'$ are the same redex, then
  $M'\equiv N$. If neither $R$ nor $R'$ is a subterm of the other,
  meaning that $R'$ is not a constituent of $M\rrtocbnnc N$, then
  immediately we have that $R$ exists in $M'$ and $R'$ exists as a CBN
  redex in $N$, and reducing either yields an $N'$ such that
  $M'\rrtocbnnc N'$ and $N\rrto N'$. It is not possible for $R$ to be
  a strict subterm of $R'$, since by definition call-by-name reduction
  does not contract strict subterms of redexes.

  In the final case, $R'$ is a strict subterm of $R$. Let $Q'$ be the
  result of contracting $R'$, and let $Q$ be the result of replacing
  the subterm $R'$ of $R$ with $Q'$. By contracting $Q$ in $M'$ we get
  an $N'$ such that $M'\rrtocbn N'$. Further, since reductions inside
  a constituent cannot cause it to no longer be strongly normalizing,
  we see that $M'\rrtocbnnc N'$.  In addition, by straightforward
  structural reasoning it is possible to conclude that the reduction
  $M\rrtocbnnc N$ replaces zero or more copies of $R'$ exactly where
  the reduction $M'\rrtocbnnc N'$ places zero or more copies of $Q'$,
  and that otherwise $N$ and $N'$ are the same, whereby we get that
  $N\rrtostar N'$.
  % FIXME: tighten up the above using a lemma about M\rrto M' and
  % N_i\rrto N_i' implies [\vec{N}/\vec{x}]M \rrto [\vec{N'}/\vec{x}]M'
\end{myproof}
\fi


%% backwards preservation of SN under CBN-NC reduction
\begin{lemma}
  \label{lemma:cbn-nc-sn}
  If $M\rrtocbnnc N$ and $\SN(N)$ then $\SN(M)$.
\end{lemma}

\iftechreport
\begin{myproof}
  We prove that $\SN(M')$ for all $M'$ such that $M\rrto M'$, whereby
  the result immediately follows, and we proceed by induction on the
  sum of $\size{N}$ and of $\size{Q_i}$ for the constituents of the
  reduction $M\rrtocbnnc N$. Using Lemma \ref{lemma:cbn-diamond}, we
  have three cases: $M'\equiv N$, where the result is immediate; $M'$
  results from reduction in a constituent of the reduction
  $M\rrtocbnnc N$, where the result follows by the induction
  hypothesis on $\size{Q_i}$ for the constituent $Q_i$ of the
  reduction; or $M'$ results from reduction not in a constituent,
  where the result follows by the induction hypothesis on $\size{N}$.
  For the latter two cases, it is straightforward to see that the
  reduction $M\rrto M'$ cannot increase $\size{N}$ or $\size{Q_i}$
  for any $Q_i$, and that both cases reduce at least one of these.
\end{myproof}
\fi


%% Standardization-y property of CBN-NC reduction
\begin{lemma}
  \label{lemma:cbn-nc-whnf}
  If $M\rrtocbnnc N$ and if $M\rrtostar Q$ for weak head normal form
  $Q$, then $N\rrtostar Q$.
\end{lemma}

\iftechreport
\begin{myproof}
  We use the weaker assumption that $M\rrtocbn N$, and proceed by
  induction on the number of steps in $M\rrtostar Q$. There is no base
  case for $M\equiv Q$, since $M\rrtocbn N$ precludes the possibility
  that $M$ is a WHNF. Thus we have $M\rrto M'\rrtostar Q$ for some
  $M'$. By Lemma \ref{lemma:cbn-diamond} we either have $M'\equiv N$,
  and the result is immediate, or $M'\rrtocbn N'$ and $N\rrtostar N'$
  for some $N'$. By the inductive hypothesis we then have that
  $N'\rrtostar Q$, and so $N\rrtostar Q$ by transitivity of
  $\rrtostar$.
\end{myproof}
\fi

\else
\fi



% %%%%%%%%%%%%%%%%%%%
% %%%%% Section %%%%%
% %%%%%%%%%%%%%%%%%%%
% \section{A Cumulative Universe Hierarchy inside ZF}
% \label{sec:model}

% In order to ensure that our interpretation is well-formed
% set-theoretically, we first define a basic ``backbone'' of our model
% by defining interpretations \Ui\ for the universes. Each \Ui\
% intuitively gives the allowed interpretations for types of sort
% $\Type_{i-1}$, or of sort \Prop\ for $i=0$. Thus each \Ui\ will
% contain triples $\triple{\Gamma}{M}{I}$ of contexts $\Gamma$, terms
% $M$ that are well-typed with respect to $\Gamma$, and valid
% interpretations $I$ of those terms. To interpret proof terms, we
% include the ``dummy'' interpretation $\dummy$, as discussed in the
% Introduction.  The dummy interpretation $\dummy$ is also used for
% ``stuck'' terms with free variables, as discussed below in Section
% \ref{subsec:interp}.


% To define the \Ui, we use transfinite recursion on ordinals. Ordinals
% and transfinite recursion are briefly reviewed here, but see any
% standard reference (e.g., Enderton \cite{enderton77}) for more
% details. The first ordinal is 0. Each ordinal $\alpha$ has a
% successor, written $\alpha+1$, which is the least ordinal greater than
% $\alpha$. Such ordinals are called \emph{successor ordinals}.
% Finally, any infinite increasing sequence of ordinals also has a least
% ordinal greater than all ordinals in the sequence. Such ordinals are
% called \emph{limit ordinals}. Following convention, we write $\lambda$
% for limit ordinals in this section. We use the natural numbers
% $1,2,\ldots$ to denote the first, second, etc.\ successors of 0; these
% are called the \emph{finite ordinals}.  The first infinite ordinal,
% written $\omega$, is a limit ordinal that is greater than all finite
% ordinals. Finally, we write $\Omega$ for the first \emph{uncountable}
% ordinal, which is an ordinal that is greater than an uncountable
% number of other ordinals. A key property of $\Omega$ is that it is
% \emph{regular}, which here means that any countable sequence of
% countable ordinals will have an upper bound which is also countable,
% i.e., less than $\Omega$.

% The \Ui\ are defined in Figure \ref{fig:universes}, using a number of
% helper definitions. The first of these, $\F(S)$, intuitively performs
% one step of constructing a universe \Ui. It takes a set $S$ and adds
% all tuples of the form $\pair{c}{I_1,\ldots, I_n}$ where each $I_i$ is
% already in $S$, as well as all countable functions on $S$, i.e., all
% functions whose domain and range are countable subsets of $S$. The set
% of countable functions on $S$ is written $\countablefun{S}{S}$.


% \begin{figure}
% \centering
% %\iftechreport\else\begin{small}\fi
% \begin{math}
% \begin{array}{@{}lcl@{}}
% \F(S) & = &
% S \;
% %\begin{array}[t]{@{}l@{}}
% %\cup\;\{\emptyset,\dummy\}
% \cup\;\setcompr{\pair{c}{\vec{I}}}{ \forall i.I_i\in S }%\\
% \;\cup\;(\countablefun{S}{S})
% %\end{array}
% \\

% \liftset{i}{S} & = &
% \setcomprlong{\triple{\Gamma}{A}{I \subseteq (\Ctxt\times\Term\times S)}}{
%   \SN(A) \mywedge \typej[\Gamma]{A}{\Type_{i-1}} \mywedge \CR{\Gamma}{A}{I}}
% \\[18pt]

% \Sii{0} & = & \Ii \;\cup\; \setcompr{S}{\exists\Gamma.\exists A.\;\triple{\Gamma}{A}{S}\in\Ui} \\
% \Sii{\alpha+1} & = & \F(\Sii{\alpha})\\
% \Sii{\lambda} & = & \bigcup_{\alpha<\lambda} \Sii{\alpha}
% \\[7pt]

% \Ii[0] & = & \{\dummy\}\\
% \Ii[i+1] & = & \Sii{\Omega}
% \\[5pt]

% \Ui & = & \liftset{i}{\Ii}
% \end{array}
% \end{math}
% %\iftechreport\else\end{small}\fi
% \caption{The Cumulative Universe Hierarchy in ZF}
% \label{fig:universes}
% \end{figure}


% The next definition in Figure \ref{fig:universes} is $\liftset{i}{S}$,
% which intuitively forms a universe from a set of allowed
% interpretations of types.  This function forms the set of all triples
% $\triple{\Gamma}{A}{I}$ such that: $I$ itself is a set of triples
% $\triple{\Gamma'}{M}{I'}$ for $I'\in S$; $A$ is strongly normalizing;
% $A$ has type $\Type_{i-1}$ (where $\Type{-1}$ denotes $\Prop$); and
% $I$ is a \emph{reducibility candidtate} for $\Gamma$ and $A$, written
% $\CR{\Gamma}{A}{I}$.  The reducibility candidates are defined below in
% Section \ref{subsec:interp}, but all we need to know here is that they
% are a definable class of sets.

% The remainder of Figure \ref{fig:universes} defines the $\Ui$ using
% the sets $\Sii{\alpha}$, defined by transfinite recursion, along with
% the sets $\Ii$.  Intuitively, $\Ii$ defines the set of all possible
% interpretations that could be used in $\Ui$, though some of these are
% removed by the $\liftset{i}{\cdot}$ function, while $\Sii{\alpha}$ is
% the set of all interpretations that can be used in $\Ui[i+1]$ after
% $\alpha$ many steps of construction. The first line defines the base
% case for $\Sii{\alpha}$ where $\alpha=0$, by combining the set $\Ii$
% of all interpretations allowed in the previous universe and the set
% of all interpretations of types in the previous universe. The
% case for successor ordinals $\alpha+1$ applies $\F$ to the previous
% step. The case for limit ordinals $\lambda$ takes the union of all
% $\Sii{\alpha}$ for $\alpha<\lambda$.  The sets $\Ii[i+1]$ are then
% defined as $\Sii{\Omega}$, i.e., as the union of all $\Sii{\alpha}$
% for countable $\alpha$.  For $i=0$, we take $\Ii[0]=\{\dummy\}$, which
% intuitively restricts the interpretations of proofs (since $\Ii[0]$ is
% used for the $\Prop$ universe) to $\dummy$.  Finally, $\Ui$ is defined
% as $\liftset{i}{\Ii}$. In the below, we also use the shorthands
% $\Ui[s]$ and $\Ii[s]$, which denote $\Ui[i+1]$ and $\Ii[i+1]$,
% respectively, when $s=\Type_i$ and denote $\Ui[0]$ and $\Ii[0]$,
% respectively, when $s=\Prop$.


% In order to better understand the above definitions, we now briefly
% consider, at a high level, how a number of constructs are interpreted
% into the \Ui.  As discussed above, a type $A$ is interpreted as a
% set $S$ of triples $\triple{\Gamma}{M}{I}$ where $\typej{M}{A}$. If
% $A$ has sort $s$, then the $I$ are all in $\Ii[s]$, and further, for
% any particular $M$, the actual interpretation of $M$ is guaranteed to
% be one of the $I$ associated with $M$ in the interpretation of $A$. We
% can take this ``one level up'' and get that the interpretation of $s$
% itself, which is $\Ui[s]$, will contain $\triple{\Gamma}{A}{S}$.
% These properties follow from the Reducibility Theorem, which
% is not proved until Section \ref{subsec:reducibility}, but it is
% useful here to understand the intent.

% Constructor applications $\ctorapp{c}{\vec{M}}{\vec{N}}$ are
% interpreted as tuples $\pair{c}{\vec{I}}$, where $\vec{I}$ are the
% interpretations of the arguments $\vec{N}$, while functions are
% interpreted as functions from the interpretations of the inputs to the
% interpretations of the corresponding outputs. Note that this
% interpretation of functions does not cause problems for functions of
% dependent types $\pitype{x}{A}{B}$, since the types $[M/x]B$ for
% different $M$ are not differentiated in the $\Ii$. The different
% $[M/x]B$ types are only differentiated when we form $\Ui[s]$.  Stated
% differently, if $A$ and $B$ both have sort $s$ (the maximum of the
% sorts of each of $A$ and $B$), then the interpretations of all $M$ of
% type $A$ and all $N$ of type $[M/x]B$ are in $\Ii[s]$. Thus we can
% show that any function $f$ from interpretations of terms $M$ of type
% $A$ to terms $N$ of type $[M/x]B$ will also be in $\Ii[s]$, since $f$
% will be added to $\Ii[s]$ by some application of the $\F$
% operator. This requires $f$ to be countable, as $\F$ only adds
% countable functions, but this holds because there are only countably
% many terms $M$ of type $A$. Our argument also requires that all
% interpretations of inputs $M$ and outputs $N$ be in $\Sii{\alpha_f}$
% for some fixed ordinal $\alpha_f<\Omega$, so that $f$ can be in
% $\Sii{\alpha_f+1}$. The latter condition follows by regularity of
% $\Omega$ (not to be confused with the Axiom of Regularity), since each
% of countably many terms $M$ or $N$ is in $\Sii{\alpha_j}$ for some
% countable ordinal $\alpha_j$.


% The use of the regularity of $\Omega$ here essentially means that we
% are relying on the fact that $\Omega$ is ``big enough'' to ensure that
% the necessary $\alpha_f<\Omega$ exists for any countable function
% $f$. (Note that this explains why use only countable functions in
% $\F$, as the regularity argument only works for countable functions.)
% At first glance, it might seem that using $\Omega$ is overkill, and
% that we might get away with a smaller ordinal. All of the natural
% numbers, of type $\nat$ (built with constructors \z\ and \s), for
% exapmle, exist in $\Sii[0]{\omega}$. Thus all functions over \nat\
% exist in $\Sii[0]{\omega+1}$, and all functions of any higher-order
% type built from $\nat$ and $\to$ (but not polymorphism) exist in
% $\Sii[0]{\omega+\omega}$. The key difficulty is in interpreting
% inductive types with constructors that can take functions that
% quantify over the inductive type itself.

% For example, the following inductive type, introduced by Coquand et
% al.\ \cite{coquand97}, captures the countable ordinals definable
% inside CiC itself:
% \[
% \begin{array}{lcl}
% \ordz & :: & \ord\\
% \ords & :: & \ord\to\ord\\
% \ordlim & :: & (\nat\to\ord)\to\ord
% \end{array}
% \]
% The constructors \ordz\ and \ords\ are used to represent zero and
% successor ordinals, respectively, while \ordlim\ forms a limit ordinal
% from an infinite sequence of ordinals, here represented as a function
% from \nat\ to \ord. Intuitively, the interpretation $\interpnod{\ord}$
% of \ord\ is the least fixed point of the set-theoretic function
% \[
% F_{\ord}(S) =
% \begin{array}[t]{@{}l@{}}
%   {\;\langle\ordz\rangle\;} \;\cup\; \setcompr{\pair{\ords}{x}}{x\in S}\\
% \;\cup\; \setcompr{\pair{\ordlim}{f}}{f\in \interpnod{\nat}\to S}
% \end{array}
% \]
% where $\interpnod{\nat}$ is the interpretation of \nat, containing
% $\langle\z\rangle$ and closed under forming $\pair{\s}{x}$ for all $x$
% in $\interpnod{\nat}$.

% As mentioned above, it is straightforward to see that all of
% $\interpnod{\nat}$ is contained in $\Sii[0]{\omega}$.  The same is
% true for elements of \ord\ built from \ordz\ and \ords.  Any function
% from (the interpretations of) \nat\ to \ord\ will not occur until
% $\Sii[0]{\omega+1}$, though, since not all of $\interpnod{\nat}$, the
% domain, occurs until $\Sii[0]{\omega}$. Thus the first \ordlim\
% interpretations will occur in $\Sii[0]{\omega+2}$. However, these
% first \ordlim\ interpretations can only use functions $f$ whose ranges
% contain only interpretations built with \ordz\ and \ords.  Whenever we
% add more \ord\ interpretations to $\Sii{\alpha}$, the next step,
% $\Sii[0]{\alpha+1}$, will contain more functions from (the
% interpretations of) \nat\ to \ord\ with the new ordinals in their
% ranges, so there will be more \ordlim\ ordinals in
% $\Sii[0]{\alpha+2}$.  Further, if we have a sequence $x_1,x_2,\ldots$
% of elements of \ord\ (which might contain \ordlim) such that each
% $x_i$ only occurs starting at $\Sii[0]{\alpha_i}$, the function $f$
% that returns $x_i$ for (the interpretation of) $i$ will first occur in
% $\Sii[0]{\alpha_{\omega}+1}$, where $\alpha_{\omega}$ is the least
% upper bound of the $\alpha_i$, and thus $\pair{\ordlim}{f}$ will first
% occur in $\Sii[0]{\alpha_{\omega}+2}$. By recursing all the way up to
% $\Omega$, however, we are guaranteed that all functions $f$ and all
% interpretations $\pair{\ordlim}{f}$ are in $\Ii[1]$.
% Note that consistency proofs of other theories of inductive sets also
% require ordinals up to $\Omega$ \cite{pohlers89}.

% The same basic approach works for inductive types
% $\ctorapp{a}{\vec{M}}{\vec{N}}$ with parameters $\vec{M}$ and type
% arguments $\vec{N}$, except that, instead of building up a set of
% interpretations, we build a function $F_a$ from interpretations of
% parameters $\vec{M}$ and type arguments $\vec{N}$ to interpretations
% of elements of $\ctorapp{a}{\vec{M}}{\vec{N}}$. A function in ZF is
% just a set of pairs, so $F_a$ can be built using least fixed-points
% just as we did for $\ord$. One potential issue is that parameters can
% be at the same universe level as $a$, meaning that, even though
% $\ctorapp{a}{\vec{M}}{\vec{N}}$ has sort $\Type_i$, the types of the
% $\vec{M}$ might have sort $\Type_{i+1}$, not $\Type_i$.  Thus the
% interpretations of $\vec{M}$ might not be in $\Ii$, even though we are
% trying to build the interpretation of $\ctorapp{a}{\vec{M}}{\vec{N}}$
% as a type in $\Ui[i+1]$. The interpretations of the elements
% $\ctorapp{c}{\vec{M}}{\vec{Q}}$ of $\ctorapp{a}{\vec{M}}{\vec{N}}$,
% however, contain only the interpretations of the arguments $\vec{Q}$,
% and not those of the parameters $\vec{M}$, so this is not a problem.


% Note, however, that this approach only works for strictly positive
% inductive types, as otherwise $F_a$ is not monotone, and so is not
% guaranteed to have a least fixed point. For example, if $a$ has
% exactly one constructor $c$ of type $(a\to a)\to a$, and if $S$ is a
% set of potential interpretations of elements of $a$ and $f$ is a
% function in $S\to S$, then $F_{a}(S)$ would contain $\pair{c}{f}$;
% however, $f$ would no longer be a function from $F_{a}(S)\to
% F_{a}(S)$, since it could not have $\pair{c}{f}$ in its domain. Viewed
% differently, a function $f$ cannot have an interpretation containing
% itself in its own domain because of the Axiom of Regularity, which
% states that set membership $\in$ is well-founded. This failure is a
% good thing, because constructors like $c$ are known to cause
% non-termination. The same argument also says that \emph{strict}
% positivity is necessary. For example, we also cannot have a
% constructor $c$ of type $((a \to B) \to B) \to a$ for some type $B$,
% since any function $f$ to which $c$ is applied would have to have
% itself in the domain of its domain. It is not a problem, though, if
% $a$ is the return type of an argument, such as for $\ordlim$, as a
% function need not contain every possible element of its return type in
% its range.


% One benefit of the Axiom of Regularity is that we can immediately
% define primitive recursion over inductive types by using well-founded
% recursion over the set membership relation $\in$ and its transitive
% closure $\in^*$.  Using the standard representation of functions and
% tuples in ZF, we have $N_i\in^*\pair{c}{\vec{N}}$ for each $N_i$, and,
% if $I$ is in the range of $N_i$ then
% $I\in^*N_i\in^*\pair{c}{\vec{N}}$. Thus, if we have set-theoretic
% formulas $\phi_0,\phi_1,\ldots,\phi_n$ that define functions, we can define a
% function
% \[
% \primrecarr[]{F}{\vec{X}\in\vec{S}}{
%   \pair{c_1}{\vec{I}}\to\phi_1(F,\vec{X},\vec{I})\;|\ldots|\\
%   \pair{c_n}{\vec{I}}\to\phi_n(F,\vec{X},\vec{I})\;|\;I\to \phi_0(\vec{X},I)
% }
% \]
% that takes $|\vec{X}|+1$ arguments and performs primitive recursion on
% its last input. The $\phi_i$ are required to always apply $F$ to
% $|\vec{X}|+1$ arguments, the last of which is an application of some
% $I_i$ to zero or more arguments.  The resulting function tests whether
% its last argument is of the form $\pair{c_i}{\vec{I}}$ for some
% $i$. If so, the function returns $\phi_i$ applied to
% $F,\vec{X},\vec{I}$, where $F$ is the whole function, used for
% recursive calls.  If the test fails, $\phi_0(\vec{X},I)$ is returned,
% where $I$ is the input being tested.  Since any recursive calls to $F$
% is guarnateed to pass a lesser argument, under the $\in^*$ relation,
% for argument $|\vec{X}|+1$, the recursion is well-founded and
% guaranteed to be well-defined.


% As a final consideration, the above discussion of interpretations
% of functions and inductive types only applies to terms which are not
% proofs, i.e., terms whose types are not of sort $\Prop$. Proofs, in
% constrast, satisfy Proof Irrelevance, meaning that their
% interpretations are always $\dummy$. Of course, we cannot know whether
% a term is a proof or not without normalizing its type.  Instead,
% this property must come from how we define interpretations.
% We handle this with the following ``smart constructors'':
% \[
% \begin{array}{@{}lcl@{}}
% \ctorstar{c}{\vec{I}} & \triangleq & \ifthen{c:A:\Prop}{\dummy}{\pair{c}{\vec{I}}}\\
% \appstar{I_1}{I_2} & \triangleq & \ifthen{I_1=\dummy}{\dummy}{I_1 (I_2)}\\
% \lamstar{x}{S'}{F(x)} & \triangleq & \ifthen{\forall x.F(x)=\dummy}{\dummy}{F}
% \end{array}
% \]
% For constructor applications, ensuring Proof Irrelevance is
% straightforward, since we can tell immediately from a constructor
% whether it belongs to an inductive type of sort $\Prop$ or not.  For
% function applications, the result is a proof iff the return type of
% the function is, which holds iff the function itself is a proof and
% has interpretation $\dummy$. For $\lambda$-abstractions, the term is a
% proof iff its result type is a proposition iff all return values have
% interpretation $\dummy$. Each of these tests are captured by the
% respective definitions above.  Finally, we also enhance
% the primitive recursion notation above, written using $\primrecname$,
% to return $\dummy$ as the recursive function when all return values
% are $\dummy$, as in the notation for $\lambda$-abstractions.
% % FIXME: say something brief about why variables need not be handled?


% \iftechreport
% \begin{lemma}
%   \label{lemma:universe-closure}
%   If $\vec{I}\in\Ii$, $S\subseteq\Ii$,
%   $F\in\countablefun{\Ii}{\Ii}$, and
%   either $i\not=0$ or $c:A:\Prop$ (i.e., $c$ is impredicative), then the
%   following are in $\Ii$:
%   \begin{enumerate}
%   \item $\ctorstar{c}{\vec{I}}$;
%   \item $\appstar{I_1}{I_2}$ when either $I_1=\dummy$ or $I_2\in\Ran(I_1)$; AND
%   \item $\lamstar{x}{S}{F(x)}$.
%   \end{enumerate}
% \end{lemma}

% \begin{myproof}
%   FIXME HERE: note that this is where we need the countable function
%   restriction

%   All three cases are immediate when $i=0$, as the only element of
%   $\Ii[0]$ is $\dummy$, so we show the result for
%   $\Ii[i+1]$ by proving that the given definitions are all in
%   $\Fomega(S)$ for $S=\proji{\bigcup\Ui}\cup\{\Ui\}$.  For part 1, if
%   $c$ is impredicative the result is also immediate. Otherwise, if
%   there is some finite $n$ such that $\vec{I}\in\F^n(S)$ then
%   $\pair{c}{\vec{I}}\in\F^{n+1}(S)\subseteq\Fomega(S)$. Otherwise, one
%   or more of the $I_i$ contain functions over $\Fomega(S)$ in their
%   transitive closures that are ``built up'' by the final clause in the
%   definition of $\F$; but the same functions are in the transitive
%   closure of $\pair{c}{\vec{I}}$ and can be built up in the same way.
%   For part 2, the result is immediate if $I_1=\dummy$.  Otherwise
%   $I_1$ must be a function that is built up using the final clause in
%   the definition of $\F$ (in some $\Ui[j]$ for $j<i$) and thus any
%   $x\in\Ran(I_1)$ must satisfy $x\in\F^{n}(S)\subseteq\Fomega(S)$. For
%   part 3, the result is immediate if $\forall x.F(x)=\dummy$ or $F\in
%   S$. Otherwise consider the sequence of steps of application of $\F$
%   used to build the elements of $\Dom(F)\cup\Ran(F)$. Since these
%   sequences of steps are all countable and since $\Dom(F)\cup\Ran(F)$
%   countable, it is straightforward to build up $F$ with a countable
%   sequence of steps of application of $\F$ that interleaves the
%   countably many countable sequences used to build
%   $\Dom(F)\cup\Ran(F)$.
% \end{myproof}
% \fi


%%%%%%%%%%%%%%%%%%%
%%%%% Section %%%%%
%%%%%%%%%%%%%%%%%%%
\section{Strong Normalization for CiC}
\label{sec:cic-sn}

In this section, we give a Uniform Logical Relations (ULR)
interpretation $\interpnod{\cdot}$ of the terms of CiC and use this to
prove SN for the well-typed terms of CiC. Types $A$ are interpreted as
sets of triples $\triple{\Gamma}{M}{I}$ with $\typej{M}{A}$ such that,
intuitively, $M$ satisfies the logical condition for $A$ and $I$ is a
valid interpretation of $M$ with respect to $A$. SN is proved via the
Reducibility Theorem, which states that
$\triple{\Gamma}{M}{\interpnod{M}}$ is in $\interpnod{A}$ whenever $M$
has type $A$ (relative to $\Gamma$) and $A$ itself satisfies
$\triple{\Gamma}{A}{\interpnod{A}}$ is in $\interpnod{s}$ for some
sort $s$. SN follows since all interpretations in $\interpnod{s}$ must
be \emph{reducibility candidates}, which among other things require
that they contain only strongly normalizing terms.


% FIXME HERE: explain that $\Gamma$ gives the allowed FVs in a term;
% move some of the discussion here from the beginning of the next
% sub-section
% - The reason for including $\Gamma$ is to allow FVs in M with no interp
% - Need to do this to prove Reducibility for lambdas


% In the remainder of this section, Section \ref{subsec:interp} gives a
% ULR interpretation for CiC that maps terms into the $\Ui$ hierarchy of
% Section \ref{sec:model}, and Section \ref{subsec:reducibility} then
% proves the Reducibility Theorem, implying SN for CiC.



%%%%%%%%%%%%%%%%%%%%%%%
%%%%% Sub-Section %%%%%
%%%%%%%%%%%%%%%%%%%%%%%
\subsection{Reducibility Candidates}
\label{subsec:reducibility-candidates}

To prove SN for the bodies of lambdas, which can
quantify over potentially empty domain types, we need to be able to support
free variables which may have no valid interpretation. To do this, we
define the concept of \emph{stuck terms}, which intuitively are the
free variables closed under application and pattern-matching
elimination forms. We then ensure (in the definition of Reducibility
Candidates, below) that all stuck terms have the dummy interpretation
$\dummy$.

\begin{definition}[Stuck Terms]
  \label{def:stuck}
  The \emph{stuck terms} are defined inductively as follows. If $M$
  is a stuck term, then so are:
  \begin{itemize}
  \item $x$ and $u$ for variable $x$ or recursive variable $u$;
  \item $M\;N$ for any $N$; and
  \item $\Mfun\;\vec{N}\;M$ for pattern-matching function $\Mfun$
    that takes $|\vec{N}|$ parameters.
  \end{itemize}
\end{definition}


Since types can also be stuck terms, we define the notation
$\denotation{\Gamma}{A}{I}$, called the \emph{set denoted by $I$}
(relative to $\Gamma$ and $A$), as follows: if $I=\dummy$,
then $\denotation{\Gamma}{A}{I}$ is the set of SN terms
of type $A$, i.e., the set
\[
\setcompr{\triple{\Gamma'}{M}{\dummy}}{\Gamma'\geq\Gamma \mywedge \typej[\Gamma']{M}{A} \mywedge \SN(M)}
\]
Otherwise $\denotation{\Gamma}{A}{I}=I$. Intuitively, this turns the
dummy interpretation into a dummy type.  We often use
$\denotationnot{I}$ when $\Gamma$ and $A$ can be inferred from
context.

We can now define the reducibility candidates, which
constitute the well-formed sets denoted by interpretations of types.
Note that these are very similar to the saturated sets of, e.g.,
Coquand and Gallier \cite{coquand90} and Geuvers \cite{geuvers94}:


\begin{definition}[Reducibility Candidate]
  We say that $I$ is a \emph{reducibility candidate for $\Gamma$ and
    $A$}, written $\CR{\Gamma}{A}{I}$ (or $\CRnot{I}$ when $\Gamma$
  and $A$ can be inferred from context), iff
  $S=\denotation{\Gamma}{A}{I}$ is a set of triples such that,
  whenever $\triple{\Gamma'}{M}{I'}\in S$, we have the following
  named properties:
  \begin{description}%\vspace{8pt}
  \item[\hfill\textbf{(G):}] \hspace*{10pt} $\Gamma'\geq\Gamma$;
  \item[\hfill\textbf{(E):}] \hspace*{10pt} If $\Gamma''\geq\Gamma'$ then $\triple{\Gamma''}{M}{I'}\in S$;
  \item[\hfill\textbf{(T):}] \hspace*{10pt} $\typej[\Gamma']{M}{A}$;

  \item[\textbf{(CR 1):}] \hspace*{10pt} $\SN(M)$;
  \item[\textbf{(CR 2):}] \hspace*{10pt} If $M\rrto M'$ then
    $\triple{\Gamma'}{M'}{I'}\in S$;
  \item[\textbf{(CR 3):}] \hspace*{10pt} If $M'\rrtocbnnc M$ then
    $\triple{\Gamma'}{M'}{I'}\in S$; and
  \item[\textbf{(CR 4):}] \hspace*{10pt} If $M'$ is a
    stuck term such that $\SN(M')$ and if $\typej[\Gamma'']{M'}{A}$
    for $\Gamma''\geq\Gamma'$ then $\triple{\Gamma''}{M'}{\dummy}\in S$.
  \end{description}
\end{definition}


As an immediate application of this definition, it is straightforward
to see that $\dummy$ defines a reducibility candidate:
\begin{lemma}
  \label{lemma:cr-dummy}
  $\CR{\Gamma}{A}{\dummy}$ for all $A$ with $\typej{A}{s}$.
\end{lemma}

\iftechreport
\begin{myproof}
  \CRG, \CRE, \CRT, \CRone, and \CRfour\ are immediate. \CRtwo\
  follows because $\SN(M)$ and $M\rrto M'$ implies $\SN(M')$.
  \CRthree\ follows by Lemma \ref{lemma:cbn-nc-sn}.
\end{myproof}
\fi


%%%%%%%%%%%%%%%%%%%%%%%
%%%%% Sub-Section %%%%%
%%%%%%%%%%%%%%%%%%%%%%%
\subsection{A ULR Interpretation for CiC}
\label{subsec:interp}

We now define the interpretation $\interp{M}$ in Figure
\ref{fig:cic-ulr}.  This is defined relative to a context $\Delta$
and an \emph{interpretation valuation} $\Xi$ of the following
syntax:
\[
\Xi ::= \cdot \bor \Xi,\dxform{N}{I} \bor \Xi,\duform{N}{I} \bor \Xi,a\mapsto I
\]
Intuitively, interpretation valuations map variables and
recursive variables to a pair of a term and an interpretation
of the term. Considering just the terms, interpretation
valuations define a substitution, written $\Dsubst$,
while $\Dlookup{\cdot}$ looks up just the interpretation associated
with some $x$, $u$, or $a$.
We also use the following abbreviated notations:
$\dgform{\vec{N}}{\vec{I}}$ represents the interpretation
valuation $\dxform[x_1]{N_1}{I_1},\ldots,\dxform[x_n]{N_n}{I_n}$; the notation
$\interpd{A}$ represents $\denotation{\Delta}{\Dsubst(A)}{\interp{A}}$;
the notations $\interp{\vec{M}}$ and
$\interp{\Gamma'}$ denote sequences of interpretations of each of the
$M_i$ or the types in $\Gamma'$, respectively; and finally
$\triple{\vec{\Delta}}{\vec{N}}{\vec{I}}\in\interpd[\iargs{\Delta_0}{\Xi}]{\Gamma}$
denotes the fact that that $\triple{\Delta_i}{N_i}{I_i}$ is in
$\interpd[\iargs{\Delta_{i-1}}{\Xi}]{A_i}$ for each type $A_i$ in
$\Gamma$.
% We also sometimes drop the context $\Gamma$ when
% it can be inferred from the text, writing $\interp[\Delta]{M}$
% instead of $\interp{M}$.


%%%%% interpretation figure %%%%%

\begin{figure*}
\centering
\iftechreport\else\begin{small}\fi
\begin{math}
\begin{array}{@{}l@{\hspace{6pt}}c@{\hspace{6pt}}l@{}}
% sorts in general
  \interp{\Type_i} & = &
  \setcompr{\triple{\Delta}{A}{I \subseteq(\Ctxt\times\Term\times\Ii)}}{
    \SN(A) \mywedge \typej[\Delta]{A}{\Type_{i-1}} \mywedge \CR{\Delta}{A}{I}}
\\

% % prop
%   \interp{\Prop} & = &
%   \setcompr{\pair{A}{I}}{ % \typejemp{A}{\Prop} \mywedge
%     \SN(A) \mywedge \CR{I} \mywedge I\in\Ui[0]}\\

% % predicative types
% \interp{\Type_i} & = &
% \setcompr{\pair{A}{I}}{ % \typejemp{A}{\Type_i} \mywedge
%   \SN(A) \mywedge \CR{I} \mywedge I\in\Ui[i+1]}\\

% Pi-types
\interp{\pitype{x}{A}{B}} & = &
\setcomprlong{\triple{\Delta\geq\Delta}{M}{(\lamstarnot{p}{\appstar{F}{p}})}}{
  \typej[\Delta']{M}{\pitype{x}{A}{B}} \mywedge\\
  %\ind{2}
  \forall(\triple{\Delta''\geq\Delta'}{N}{I}\in\interpd[\iargs{\Delta'}{\Xi}]{A}) .\;
  \triple{\Delta''}{M\;N}{\appstar{F}{\triple{\Delta''}{N}{I}}} \in\interpd[\iargs{\Delta''}{\Xi,\dxform{N}{I}}]{B}
}
\\
%\\

% inductive types
\interp{\ctorapp{a}{\vec{M}}{\vec{N}}} & = &
(\ifthen{a\mapsto F\in\Xi}{F}{\interpind{a}})
(\Delta,\Dsubst(\vec{M}),\Dsubst(\vec{N}), \interp{\vec{M}},\interp{\vec{N}})

\\
%\\

% bottom
%\interp{\bot} & = & \dummy
%\\

% variables
\interp{x} & = & \Dlookup{x}
\\

% recursive variables
\interp{u} & = & \Dlookup{u}
\\

% applications
\interp{M\;N} & = & \appstar{\interp{M}}{\triple{\Delta}{N}{\interp{N}}}\\

% lambdas
\interp{\lamabs{x}{A}{M}} & = &
\lamstar{\triple{\Delta'}{N}{I}}{\interpd{A}}{\interp[\iargs{\Delta'}{\Xi,\dxform{N}{I}}]{M}}
\\

% constructors
% \interp{\ctorapp{c^{-1}}{\vec{M}}{\vec{N}}} & = & \dummy\\
% \interp{\ctorapp{c^i}{\vec{M}}{\vec{N}}} & = & \pair{c}{\interp{\vec{N}}}\\
\interp{\ctorapp{c}{\vec{M}}{\vec{N}}} & = &
%\ifthen{c:\Prop}{\dummy}{\pair{c}{\interp{\vec{N}}}}
\ctorstar{c}{\interp{\vec{N}}}
\\[5pt]

% predicative pm-functions
%\\
\multicolumn{3}{@{}l@{}}{
\interp{\Mfun=\ppmfun{u}{\Gammaarg}{\pc{\vec{c}}{\vec{\Gamma}}{\vec{M}}}}
}
\\
& = &
\begin{array}[t]{@{}l@{}}
  \primrecnl{F}{\triple{\vec{\Delta}}{\vec{N}}{\vec{I}}\in\interpd{\Gammaarg}}{
        \ldots|\;\triple{\Delta''}{\ctorapp{c_i}{\vec{Q}}{\vec{N'}}}{\pair{c_i}{\vec{I'}}} \to \interp[\iargs{\Delta''}{\Xi,\dgargform{\Gamma}{\vec{N}}{\vec{I}},\dgargform{\Gamma_i}{\vec{N'}}{\vec{I'}},\duform{\Mfun}{F}}]{M_i}
        \;|\ldots|\;X \to \dummy
      }
  \end{array}
% \primrec{F}{
%   \begin{array}[t]{@{}l@{}}
%     \lamstar{\pair{\vec{N}}{\vec{I}}}{\interpd{\Gammaarg}}{
%       \lamstar{\pair{N}{I}}{\interpd[\Delta,\Gammaarg\mapsto\pair{\vec{N}}{\vec{I}}]{\ctorapp{a}{\vec{Q}}{\vec{R}}}}{\\
%         \ind{1} \mathsf{match}\;\pair{\NF(N)}{I}\;\mathsf{with}\\
%         \ind{2}|\;
%         \pair{\ctorapp{c_i}{\vec{v}}{\vec{v'}}}{\pair{c_i}{\vec{I'}}} \to \interp[\Delta,\Gammaarg\mapsto\pair{\vec{N}}{\vec{I}},\Gamma_i\mapsto\pair{\vec{v'}}{\vec{I'}},\duform{F}]{M_i}
%         \\
%         \ind{2}|\;\pair{v}{I} \to \dummy
%       }}
%   \end{array}
% }
\\[5pt]

% impredicative pm-functions
%\\
\multicolumn{3}{@{}l@{}}{
\interp{\Mfun=\ipmfun{u}{\Gammaarg}{\pc{\vec{c}}{\vec{\Gamma}}{\vec{M}}}}
}
\\
& = &
\begin{array}[t]{@{}l@{}}
  \lamstar{\triple{\vec{\Delta}}{\vec{N}}{\vec{I}}}{\interpd{\Gammaarg}}{\lamstar{I}{\interpd[\iargs{\Delta_n}{\Xi}]{\ctorapp{a}{\vec{Q}}{\vec{R}}}}{
    \\\ind{1}
    \bigsqcap_{\triple{\Delta''}{\vec{N'}}{\vec{I'}}\in\interpd{\Gamma_i,\Afun}}\interp[\iargs{\Delta''}{\Xi,\dgform[\Gammaarg]{\vec{N}}{\vec{I}},\dgform[(\Gamma_i,u)]{\vec{N'}}{\vec{I'}}}]{M_i}
    }}
  \end{array}

\\ \\

\multicolumn{3}{@{}l@{}}{
\interpind{a} %\hspace{35pt}
=
\begin{array}[t]{@{}l@{}}
\mu F (\Delta, \vec{M}, \vec{N}, \vec{X},\vec{Y}).\\
\ind{1}
\setcomprarr{\triple{\Delta'}{M}{I}}{
  % typing and SN
  \Delta'\geq\Delta \mywedge \SN(M) \mywedge
  \typej[\Delta']{M}{\ctorapp{a}{\vec{M}}{\vec{N}}} \mywedge
  \\

  % arguments are in their LRs
  (%\begin{array}[t]{@{}l@{}}
    \forall\vec{Q}\forall\vec{R}.\; M\rrtostar\ctorapp{c}{\vec{Q}}{\vec{R}}
    \myimplies c:\pigamma{\Gamma_p}{\pigamma{(x_1\!:\!A_1,\ldots,x_n\!:\!A_n)}{\ctorapp{a}{\Gamma_p}{\vec{M_c}}}} \in\Sigma
    \mywedge\\
    \ind{2}
    \exists \Delta''.\exists\vec{I}.\forall i.\;
    \triple{\Delta''}{R_i}{I_i}\in\interpd[a\mapsto F,\Gammap\mapsto\vec{X},\vec{x}\mapsto\vec{I}]{A_i}
    \mywedge I=\ctorstar{c}{\vec{I}}
    \mywedge \interp[\Gammap\mapsto\vec{X},\vec{x}\mapsto\vec{I}]{\vec{M_c}}=\vec{Y}) \mywedge
  %\end{array}
  \\
  %\iftechreport\else\\\ind{1}\fi
%   (a\not\,:\pigamma{\Gammap}{\pigamma{\Gammaa}{\Prop}} \myimplies I=\pair{c}{\vec{I}})
%   \mywedge
%  \\

  % special forms of I
  (\neg(\exists c.\exists\vec{Q}\exists\vec{R}.\;M\rrtostar\ctorapp{c}{\vec{Q}}{\vec{R}})
  \myimplies I=\dummy)

%   \left(\begin{array}{l}
%       (I=\pair{c}{\vec{I'}} \mywedge \NF(M)=\ctorapp{c}{\vec{v'}}{\vec{v}}
%       \mywedge \pair{(\vec{v},\vec{v'})}{(\vec{I},\vec{I'})}\in\interpd{\Gammac})
%       \\
%       \vee
%       (I=\dummy \mywedge a : \pigamma{\Gammaa}{\Prop})
%       \vee
%       (I=\dummy \mywedge \not\exists c.\NF(M)=\ctorapp{c}{\vec{v'}}{\vec{v}})
%   \end{array}\right)
}
\end{array}
}

\end{array}
\end{math}
\iftechreport\else\end{small}\fi
\caption{A Uniform Logical Relations Interpretation for CiC}
\label{fig:cic-ulr}
\end{figure*}


Let $\Term(i)$ denote the set of all terms that do not contain
$\Type_{j}$ for $j\geq i$, denoting $\Prop$ by $\Type_{-1}$.  The
interpretation $\interp{M}$ is defined by primary induction the least
$i$ such that $M\in\Term(i)$, by secondary induction on the number of
inductive types $a$ \emph{not} in $\Dom(\Delta)$, and by tertiary
induction on the structure of $M$. The primary induction hypothesis
used to define the interpretation of sorts $\Type_i$, which works
because normal forms of type $A$ such that $A$ has type $\Type_i$ do
not contain $\Type_i$. The secondary induction hypothesis is used to
define the interpretations of inductive types, which works because the
type of any $a$ can only contain constructors and type constructors
which come before $a$ in $\Sigma$.  Otherwise $\interp{\cdot}$ is
defined using the tertiary induction hypothesis, structural induction
on $M$.

In order to handle the dummy interpretation $\dummy$, we define the
following ``smart constructors'':
\[
\begin{array}{@{}lc@{\hspace{5pt}}l@{}}
\ctorstar{c}{\vec{I}} & \triangleq & \ifthen{c:A:\Prop}{\dummy}{\pair{c}{\vec{I}}}\\
\appstar{F}{I} & \triangleq & \ifthen{F=\dummy}{\dummy}{F (I)}\\
\multicolumn{3}{@{}l@{}}{
  \lamstar{x}{S}{F(x)} \triangleq \ifthen{\forall x.F(x)=\dummy}{\dummy}{F}
}
\end{array}
\]
The first constructs the interpretation of
$\ctorapp{c}{\vec{M}}{\vec{N}}$ from the interpretations $\vec{I}$ of
$\vec{N}$. In order to satisfy Proof Irrelevance, the result is
$\dummy$ when $c$ is impredicative, and otherwise is
$\pair{c}{\vec{I}}$. The second applies a function $F$ to an argument
$I$, returning $\dummy$ if $F$ is $\dummy$.  The third constructs a
set-theoretic function using $\lambda$ notation, returning $\dummy$ if
the function is uniformly $\dummy$.


As discussed above, a type $A$ is interpreted as a set of triples
$\triple{\Delta}{M}{I}$ where $\typej{M}{A}$.  Thus the interpretation
of $\Type_i$, which is a type of types, contains all possible triples
$\triple{\Delta}{A}{I}$ where $A$ is a type and $I$ itself is a set of
triples that forms a reducibility candidate for $A$. In order to
include all possible reducibility candidates $I$ for $A$, we
quantify $I$ over the set of all triples $\triple{\Delta'}{M}{I'}$
where $I'$ is in $\Ii$, defined as follows:
\[
\begin{array}{@{}l@{\hspace{5pt}}c@{\hspace{5pt}}l@{}}
  \Ii[-1] & = & \{*\}\\
  \Ii[i\geq 0] & = & \mu X.\setcomprarr{\interp{M}}{M\in\Term(i) \mywedge \\\Ran(\Xi)\subseteq \Term(i)\times X}
\end{array}
\]
$\Ii[-1]$ contains only $*$ because of Proof Irrelevance, while all
other $\Ii$ contain the set of all interpretations of any $M$ that
does not itself contain $\Type_i$.


The interpretation of function types $\pitype{x}{A}{B}$ is the set of
all triples $\triple{\Delta'}{M}{F}$ such that, for all
$\triple{\Delta''}{N}{I}$ in $\interpd[\iargs{\Delta'}{\Xi}]{A}$ we
have that applying $M$ to $N$ and applying $F$ to
$\triple{\Delta''}{N}{I}$ yields a triple (with first element
$\Delta''$) in the interpretation of $B$.  Note that we require $F$ to
have the form $\lamstarnot{p}{\appstar{F}{p}}$, meaning that
$F=\dummy$ whenever all of the outputs of $F$ are $\dummy$.

Inductive type constructors $a$ are interpreted by applying either
$\Dlookup{a}$, if it is defined, or the result of the helper
definition $\interpind{a}$, to the parameters and type
arguments, along with their interpretations, of $a$.
The helper definition constructs a function $F$
using the least fixed-point operator $\mu$, returning a set
of triples $\triple{\Delta'}{M}{I}$ where $M$ is strongly normalizing
of the correct type and that satisfies the following: if $M$ rewrites
to a constructor term $\ctorapp{c}{\vec{Q}}{\vec{R}}$, then $I$ must
be $\ctorstar{c}{\vec{I}}$, where the $\vec{I}$ are valid
interpretations for the $\vec{R}$, i.e., the $\vec{R}$ occur in
triples with the $\vec{I}$ in the interpretations of their types. Note
that, if $a$ is impredicative, $\ctorstar{c}{\vec{I}}=\dummy$.
Further, we require $c$ to be a constructor of $a$ and the
interpretations of the arguments $\vec{M_c}$ in the return type of $c$
to equal the arguments $\vec{Y}$. Otherwise, if $M$ rewrites to no
such term, $I=\dummy$. We can see that this use of the least
fixed-point operator $\mu$ is guaranteed to be well-defined because
$a$ can only be used strictly positively in the argument types $A_i$
of its constructors.


The interpretations of variables and recursive variables look those
variables up in $\Delta$. The interpretation of an application applies
(using $\appstar{I_1}{I_2}$) the interpretation of the function to
that of the argument. The interpretation of a $\lambda$-abstraction
forms a function (using the $\lamstar{I}{S}{F(x)}$ notation) that
takes in an interpretation for its argument and returns the
interpretation of the body. The interpretation of a constructor term
$\ctorapp{c}{\vec{M}}{\vec{N}}$ applies the constructor (using the
$\ctorstar{c}{\vec{I}}$ notation) to the interpretations of the
arguments $\vec{N}$.

The interpretation of a predicative pattern-matching function performs
primitive recursion on the scrutinee. This is
justified by the Axiom of Regularity, since any recursive calls are
through interpretations of the recursive variable $u$, which is
guaranteed to only be applied to variables that match subterms of the
scrutinee. If the scrutinee is a triple of a context $\Delta''$, an
application of $c_i$, and the interpretation $\pair{c_i}{\vec{I}}$ of
an application of $c_i$, then the interpretation of $M_i$ is returned,
possibly with primitive recursive calls to the entire function;
otherwise, $\dummy$ is returned in the catch-all case, which holds when
the scrutinee is a stuck term.


The final case is that for impredicative pattern-matching functions.
Because of Proof Irrelevance, the scrutinee for such a function will
always have interpretation $\dummy$, and thus we cannot perform
primitive recursion on it. Instead, the interpretation of an
impredicative pattern-matching function throws away its input, and
instead considers \emph{all possible} interpretations of the pattern
cases $M_i$, for all possible interpretations that can be assigned to
the pattern variables $\Gamma_i$ and the recursive function variable
$u$.  If all possible interpretations of the $M_i$ are all
equal to some result $I$, then the interpretation of the impredicative
pattern-matching function returns $I$. Otherwise the interpretation is
undefined. This is written with the notation $\bigsqcap S$, where
$\bigsqcap\{I\}=I$, $\bigsqcap\{\}=\dummy$, and $\bigsqcap S$ is
undefined otherwise.

The reason this approach works is due to Proof Irrelevance. If the
return type is a proposition, then all pattern cases are proofs and so
must all have the same interpretation $\dummy$. Otherwise we either
have that the scrutinee type must be an empty type, so there are no
pattern cases (and $\dummy$ is returned), or the scrutinee type must
be a singleton type whose sole constructor takes only proof arguments,
and there is only one possible interpretation, again $\dummy$, for
those arguments.
% Note that one ramification of this approach to
% interpreting impredicative pattern-matching functions is that
% $\interp{M}$ is not always defined for ill-typed $M$.


The following structural properties can be proved by straightforward
induction on the definition of $\interp{\cdot}$:

\begin{lemma}[Interpretation Weakening]
  \label{lemma:weakening}
  $\interp[\iargs{\Delta}{\Xi_1,\Xi_2,\Xi_3}]{M}=\interp[\iargs{\Delta}{\Xi_1,\Xi_3}]{M}$
  for any $M$ and $\Xi_i$ when both of these are defined.
\end{lemma}


\begin{lemma}[Interpretation Substitution]
  \label{lemma:substitution}
  $\interp[\iargs{\Delta}{\Xi,\dxform{N}{\interp{N}}},\Xi']{M}=
  \interp[\iargs{\Delta}{\Xi,\Xi'}]{[N/x]M}$ or both of these are
  undefined. % for any $M$, $N$, $\Delta$, $\Xi$, and $\Xi'$.
\end{lemma}



%%%%%%%%%%%%%%%%%%%%%%%
%%%%% Sub-Section %%%%%
%%%%%%%%%%%%%%%%%%%%%%%
\subsection{Proving Reducibility}
\label{subsec:reducibility}

In order to prove Reducibility, we first need to define the notion of
well-formed interpretatino valuations.
We say that $\Xi$ is \emph{well-formed} with respect to input
context $\Gamma$ and output context $\Delta$,
written $\ivtypej{\Xi}{\Delta}$, iff:
\begin{enumerate}
\item $\typej{\Dsubst[\Xi]}{\Delta}$, meaning that
  $\Dom(\Dsubst[\Xi])=\Dom(\Gamma)$ and that
  $\forall x\in\Dom(\Dsubst[\Xi]).\;\typej[\Delta]{\Dsubst[\Xi](x)}{\Dsubst[\Xi](\Gamma(x))}$;

\item $\Dom(\Xi)$ restricted to variables and recursive variables
  equals $\Dom(\Gamma)$;

\item For all $x\in\Dom(\Gamma)$ we have
  $\triple{\Delta}{\Dsubst[\Xi](x)}{\Dlookup{x}}\in\interpd{\Gamma(x)}$;

\item For all $\guform{n}{\vec{x}}{\pigamma{\Gamma_u}{B}}\in\Gamma$ and for all $x_i$ we have
  $\triple{\Delta}{\Dsubst[\Xi](u)\;\vec{N}\;\Dsubst[\Xi](x_i)}{\appstar{\appstar{\Dlookup{u}}{\vec{I}}}{\Dlookup{x_i}}}\in\interpd{B}$
  whenever $\forall j.\;\triple{\Delta}{N_j}{I_j}\in\interpd{A_j}$ where
  $A_1,\ldots,A_m$ are the types in $\Gamma_u$; and

\item For all $a\in\Dom(\Xi)$ we
  have that $\Dlookup{a}$ is a function from
  $(\Delta,\vec{M},\vec{N},\vec{I},\vec{I'})$ for
  $\triple{\Delta}{\vec{M},\vec{N}}{\vec{I},\vec{I'}}\in\interpd{\Gamma_p,\Gamma_a}$
  to reducibility candidates for $\ctorapp{a}{\vec{M}}{\vec{N}}$.
\end{enumerate}
At a high level, the idea is that, whenever we form $\interp{M}$,
$\Gamma$ lists the function arguments in scope, and $\Xi$ lists terms
and interpretations that have been passed in for these arguments.  In
order to state that an interpretation passed for $x$ is valid, it must
be associated with the term $\Dsubst[\Xi](x)$ passed for $x$.  The
output context $\Delta$ means that the range of $\Dsubst[\Xi]$ can
contain open terms. For recursive variables $u$, the requirement is
looser, only requiring that the interpretation of any fully applied
occurrence of $u$ is valid given the value of $\Dlookup{u}$.  As an
example, for any $\Gamma$ we have
$\ivtypej{\dgform[\Gamma]{\Gamma}{\vec{\dummy}}}{\Gamma}$ where
$\dgform[\Gamma]{\Gamma}{\vec{\dummy}}$ assigns term $x$ and
itnerpretation $\dummy$ to each variable $x\in\Dom(\Gamma)$.  Using
$\dummy$ here is valid by \CRfour\ because the variables of $\Gamma$
and fully applied recursive variables of $\Gamma$ are stuck terms.


\begin{lemma}[Proof Irrelevance]
  \label{lemma:proof-irrelevance}
  Let $\ivtypej{\Xi}{\Delta}$ and
  $\triple{\Delta'}{\Dsubst{B}}{\interp{B}}\in\interpd[]{\Prop}$.  If
  $\triple{\Delta''}{M}{I}\in\interpd{B}$ then $I=\dummy$.
\end{lemma}

\iftechreport
FIXME HERE: proof!
\fi

% FIXME HERE: tweak this explanation of impred pm funs a bit...
Conversion is mostly straightforward by Interpretation
Substitution. The interesting case is reduction of an impredicative
pattern-matching function applied to a constructor application in the
scrutinee. The result follows in this case from the requirement that
the interpretation of the source term $M$ is well-defined, as this
ensures that the interpretation of the result, which has the form
$\sigma M_i$ for some $\sigma$ and some pattern-case $M_i$ of the
impredicative pattern-matching function, is independent of the
particular $\sigma$ used, and so is equal to the interpretation
returned by the pattern-matching function. Subtyping then follows
directly.


\begin{lemma}[Conversion]
  \label{lemma:interp-eq}
  If $M\rrto N$ then we have that $\interp{M}=\interp{N}$ whenever both of these
  are defined.
\end{lemma}

\iftechreport
\begin{myproof}
  By straightforward induction on $M$.
\end{myproof}
\fi


%% subtyping
\begin{lemma}[Subtyping]
  \label{lemma:subtyping}
  If $A_1\subtype A_2$ then we have that $\interpd{A_1}\subseteq\interpd{A_2}$.
\end{lemma}

\iftechreport
\begin{myproof}
  By straightforward induction, using the fact that
  $\interpd{s_1}\subseteq\interpd{s_2}$ whenever $s_1\subtype s_2$ for
  the base case.
\end{myproof}
\fi


We now prove the main result of this section, Reducibility, which
states that if $\typej{M}{A}$ then $\interp{M}$ is well-defined and is
a valid interpretation of $M$.  This assumes that $\interp{A}$ itself
is valid, meaning that it is a valid reducibility candidate. As
discussed above, we must also assume $\ivtypej{\Xi}{\Delta}$.  Since,
at a high level, $\Xi$ acts as an ``interpretation substitution'' for
the variables in $M$, we have that $\interp{M}$ is actually a valid
interpretation for $\Dsubst(M)$, with respect to the output context
$\Delta$.  Thus Reducibility yields
$\triple{\Delta}{\Dsubst(M)}{\interp{M}}\in\interpd{A}$.  This is not
a restriction, though, as we can always take $\Delta=\Gamma$ and
$\Xi=\dgform[\Gamma]{\Gamma}{\vec{\dummy}}$ as discussed above,
which can be used to show SN for terms $M$ with free variables.

%%%%% Reducibility Theorem %%%%%

\begin{theorem}[Reducibility]
  If $\typej{M}{A}$ and $\typej{A}{s}$, and if
  $\triple{\Delta}{\Dsubst(A)}{\interp{A}}\in\interpd[]{s}$ for
  $\ivtypej{\Xi}{\Delta}$, then
  $\interp{M}$ is defined and
  $\triple{\Delta}{\Dsubst(M)}{\interp{M}}\in\interpd{A}$.
\end{theorem}

\begin{myproof}
%%%%%%%%%%%%%%%%%%%%%%%%%%%%%%%%%%%%%%%%%%%%%%%%%%%%%%%%%%%%%%%%%%%%%%
%%%%% Tech Report Proof Details                                  %%%%%
%%%%%%%%%%%%%%%%%%%%%%%%%%%%%%%%%%%%%%%%%%%%%%%%%%%%%%%%%%%%%%%%%%%%%%
\iftechreport
FIXME HERE: fix up the proofs!
\begin{itemize}
\item we are now using triples of typed terms
\item ``meaningless terms'' $\longrightarrow$ ``stuck terms''
\end{itemize}


FIXME HERE: check this paragraph\ldots

The following lemmas now prove each of the cases of the Reducibility
Theorem. An interesting aspect of this proof is that we do not have to
prove, as a separate lemma, that $\CR{\Gamma}{A}{\interpd{A}}$ holds, i.e., that the
interpretation of $A$ gives a valid Reducibility Candidate; this result is part of Reducibility for
$A$, i.e., it is implied by
$\triple{\Gamma}{A}{\interp{A}}\in\interpd{s}$ for $\typej{A}{s}$.  In the
below, we say that term $M$ is reducible at $\Gamma$ and $A$, relative to
$\Delta$ and $\sigma$, iff $\tctypej$ and the triple
$\triple{\Gamma'}{\sigma(M)}{\interp{M}}$ is in $\interpd{A}$. We often omit $A$,
$\Delta$, and/or $\sigma$ when these are clear from context. In
addition, we say that $\Gamma=x_1:A_1,\ldots,x_n:A_n$ is reducible
relative to $(\Delta;\sigma)$ iff $\tctypejdef$ and, for all $\Delta_i$
and $\sigma_i$ that map variables $x_j$ for $j<i$ to terms
$\pair{N_j}{I_j}\in\interpd[\Delta_i]{A_j}$ we have that
$\pair{(\sigma,\sigma_i)(A_i)}{\interp[\Delta,\Delta_i]{A_i}}\in\interpd[]{s_i}$
for some $s_i$. Again, we often omit $\Delta$ and $\sigma$ when they
are clear from context.  \iftechreport Finally, we often appeal to
properties \CRone\ through \CRfour\ of a type $A$, where technically
what is meant is the properties of the set $\interpd{A}$ for the
appropriate $\Delta$.  \fi


%% sorts
\begin{lemma}[Inclusion of Sorts]
  \label{lemma:sorts}
  If $\typejemp{s_1}{s_2}$ then we have that
  $\triple{\Gamma}{s_1}{\interp{s_1}}\in\interpd[]{s_2}$ for all
  $\Gamma$ and $\Delta$.
\end{lemma}

\iftechreport
\begin{myproof}
  $\SN(s_1)$ is immediate, as is $\pair{s_1}{\interp{s_1}}\in\Ui$
  where $s_2=\Type_i$ ($s_2$ cannot be $\Prop$).  All that remains is
  to prove $\CRnot{\interp{s_1}}$:
  \begin{description}
  \item[\textbf{(CR 1):}] \hspace*{10pt} Immediate.
  \item[\textbf{(CR 2):}] \hspace*{10pt} If $\SN(A)$ and $A\rrto A'$,
    then $\SN(A')$; the remaining conditions are immediate.
%     and $\CR{A'}{I}$ holds using the conversion typing rule.  For
%     $s_1=\Prop$, we also have
%     $\denotation{I}=\denotation{\Delta(A')}{I}$, also by
%     the conversion typing rule, so the final condition for membership
%     in $\interp[]{\Prop}$ also holds.
  \item[\textbf{(CR 3):}] \hspace*{10pt} For \CRthree, if $\SN(A)$ and
    $A'\rrtocbnnc A$ then $\SN(A')$ holds by Lemma
    \ref{lemma:cbn-nc-sn}; the remaining conditions are immediate.
%     $\CR{A}{I}$ hold, and if $A'\rrtocbnnc A$, then $\SN(A')$ holds by
%     Lemma \ref{lemma:cbn-nc-sn} and $\CR{A'}{I}$ holds by the
%     conversion typing rule.
  \item[\textbf{(CR 4):}] \hspace*{10pt} If $A$ is a meaningless term
    and $\SN(A)$, then $\CRnot{\dummy}$ holds by Lemma
    \ref{lemma:cr-dummy} and $\dummy\in\Ui[0]\subseteq\Ui$ is
    immediate.
  \end{description}
\end{myproof}
\fi


%% pi types
\begin{lemma}
  \label{lemma:pi}
  Let $s_B\subtype s$ and either $s_B=\Prop$ or $s_A\subtype s$.
  Further let $\tctypejdef$. If
  $\pair{\sigma(A)}{\interp{A}}\in\interpd{s_A}$ and for all
  $\pair{N}{I}\in\interpd{A}$ we have
  $\pair{\sigma([N/x]B)}{\interp[\Delta,\dxform{I}]{B}}\in\interpd[]{s_B}$
  then
  $\pair{\sigma(\pitype{x}{A}{B})}{\interp{\pitype{x}{A}{B}}}\in\interpd[]{s}$.
\end{lemma}

\iftechreport
\begin{myproof}
  We have $\SN(\sigma(A))$ by definition of $\interpd[]{s_A}$. By
  \CRfour\ for $A$ we have $\pair{\bot}{\dummy}\in\interpd{A}$, so by
  assumption and by definition of $\interpd[]{s_B}$ we have and
  $\SN([\sigma,\bot/x]B)$, and thus we have $\SN(B)$.  Hence we have
  $\SN(\pitype{x}{A}{B})$.

  We now show $\interp{\pitype{x}{A}{B}}\in\Ui$ for $s=\Type_i$ or for
  $i=0$ and $s=\Prop$.  If $s_B=\Prop$ then, since by assumption $B$
  is reducible at $\Prop$, we must have that
  $\appstar{F}{\pair{N}{I}}=\dummy$ for all
  $\pair{M}{F}\in\interp{\pitype{x}{A}{B}}$.  Thus
  $\lamstarnot{p}{\appstar{F}{p}}=\dummy$ and so $F=\dummy$ by
  definition, and we have
  $\interp{\pitype{x}{A}{B}}\in\Ui[0]\subseteq\Ui$.  Otherwise by
  assumption we have both $s_A\subtype s$ and $s_B\subtype s$, and so
  by assumption and the definition of $\interpd[]{s_A}$ and
  $\interpd[]{s_B}$ we have $\interpd{A}\in\Ui[j]$ and
  $\interpd[\Delta,\dxform{I}]{B}\in\Ui[k]$ for all
  $\pair{N}{I}\in\interpd{A}$, where $j,k\leq i$. Hence for all
  $\pair{M}{F}\in\interpd{\pitype{x}{A}{B}}$ we have that $F$ is a
  function whose domain and range is in $\Ui$, and the result is
  immediate.

  Finally, we show that $\CRnot{\interp{\pitype{x}{A}{B}}}$ holds:
  \begin{description}
  \item[\textbf{(CR 1):}] \hspace*{10pt} By \CRfour, we have
    $\pair{\bot}{\dummy}\in\interpd{A}$, so if
    $\pair{M}{F}\in\interp{\pitype{x}{A}{B}}$ then we have
    $\pair{M\;\bot}{\appstar{F}{\dummy}}
    \in\interpd[\Delta,\dxform{\dummy}]{B}$.  By \CRone\
    for $B$ we have $\SN(M\;\bot)$ and hence $\SN(M)$.
  \item[\textbf{(CR 2):}] \hspace*{10pt} If
    $\pair{M}{F}\in\interpd{\pitype{x}{A}{B}}$ and $M\rrto M'$ then
    for all $\pair{N}{I}\in\interpd{A}$ we have that
    $\pair{M\;N}{\appstar{F}{\pair{N}{I}}}\in
    \interp[\Delta,\dxform{I}]{B}$. By \CRtwo\ for $B$ we have
    $\pair{M'\;N}{\appstar{F}{\pair{N}{I}}}\in
    \interp[\Delta,\dxform{I}]{B}$, so
    $\pair{M'}{F}\in\interp{\pitype{x}{A}{B}}$.
  \item[\textbf{(CR 3):}] \hspace*{10pt} If $M'\rrtocbnnc M$ and
    $\pair{M}{F}\in\interp{\pitype{x}{A}{B}}$, then for all
    $\pair{N}{I}\in\interpd{A}$ we have that
    $\pair{M\;N}{\appstar{F}{I}}\in\interpd[\Delta,\dxform{I}]{B}$.
    It also follows that $M'\;N\rrtocbnnc M\;N$ by Lemma
    \ref{lemma:cbn-nc-app}, and thus by \CRthree\ for $B$ we have
    $\pair{M'\;N}{\appstar{F}{I}}\in\interpd[\Delta,\dxform{I}]{B}$,
    and the desired result is immediate.
  \item[\textbf{(CR 4):}] \hspace*{10pt} Let $M'$ be a meaningless
    term such that $\SN(M')$. For all $\pair{N}{I}\in\interpd{A}$ we
    have $\SN(N)$ by \CRone\ for $A$, and so $M'\;N$ is a meaningless
    term with $\SN(M'\;N)$ by Lemma
    \ref{lemma:meaningless-app-sn}. Thus
    $\pair{M'\;N}{\dummy}\in\interpd[\Delta,\dxform{I}]{B}$ holds by
    \CRfour\ for $B$. Since $\dummy=\appstar{\dummy}{I}$, the result
    follows.
  \end{description}
\end{myproof}
\fi


%% inductive types
\begin{lemma}
  \label{lemma:ind-types}
  Let $a:\pigamma{\Gammap}{\pigamma{\Gammaa}{s}}$, and let the context
  $\Gammap,\Gammaa$, as well as all contexts $\Gammap,\Gammac$ for
  constructors $c$ of $a$, be reducible relative to $\cdot$.  If
  $\pair{\sigma(\vec{M},\vec{N})}{\interp{\vec{M},\vec{N}}}\in\interpd{\Gammap,\Gammaa}$
  then
  $\pair{\sigma(\ctorapp{a}{\vec{M}}{\vec{N}})}{\interp{\ctorapp{a}{\vec{M}}{\vec{N}}}}\in
  \interp{s}$.
\end{lemma}

\iftechreport
\begin{myproof}
  By \CRone\ for $\Gammap,\Gammaa$ we have $\SN(\sigma(M_i))$ and
  $\SN(\sigma(N_j))$ for all $i$ and $j$, and so
  $\SN(\sigma(\ctorapp{a}{\vec{M}}{\vec{N}}))$. Also, if $s=\Prop$,
  then we immediately have $I=\dummy$ for all
  $\pair{M}{I}\in\interpd{\ctorapp{a}{\vec{M}}{\vec{N}}}$, so
  $\interp{\ctorapp{a}{\vec{M}}{\vec{N}}}\in\Ui[0]$.  Otherwise
  $s=\Type_i$, and the well-formedness rules for inductive types (see
  Section \ref{sec:cic}) require that all of the types in $\Gammac$
  for any constructor $c$ of $a$ to have type $s$. Thus if
  $I=\pair{c}{\vec{I}}$ we have that each $I_j\in S_j$ for some
  $S_j\in\Ui$, and it then follows that
  $\interp{\ctorapp{a}{\vec{M}}{\vec{N}}}\in\Ui[i]$.

  What remains is to prove
  $\CRnot{\interpd{\ctorapp{a}{\vec{M}}{\vec{N}}}}$:
  \begin{description}
  \item[\textbf{(CR 1):}] \hspace*{10pt} Immediate.
  \item[\textbf{(CR 2):}] \hspace*{10pt} Straightforward, as reduction
    preserves $\SN$ and normal forms and because, if
    $M'\rrtostar\ctorapp{c}{\vec{Q}}{\vec{R}}$ for $M\rrto M'$ then
    $M\rrtostar\ctorapp{c}{\vec{Q}}{\vec{R}}$.
  \item[\textbf{(CR 3):}] \hspace*{10pt} Assume that $M'\rrtocbnnc M$
    and that $\pair{M}{I}\in\interpd{\ctorapp{a}{\vec{M}}{\vec{N}}}$.
    By Lemma \ref{lemma:cbn-nc-sn}, we have $\SN(M')$. By Lemma
    \ref{lemma:cbn-nc-whnf}, we have that
    $M'\rrtostar\ctorapp{c}{\vec{Q}}{\vec{R}}$ implies
    $M\rrtostar\ctorapp{c}{\vec{Q}}{\vec{R}}$, so the condition
    \[
    \exists\vec{I}\exists\vec{I'}.\;
    \pair{(\vec{Q},\vec{R})}{(\vec{I},\vec{I'})}\in\interpd{\Gammac})
    \]
    holds because
    $\pair{M}{I}\in\interpd{\ctorapp{a}{\vec{M}}{\vec{N}}}$.  The
    remaining conditions are straightforward because $M'$ has the same
    normal form as $M$.
  \item[\textbf{(CR 4):}] \hspace*{10pt} Immediate because any
    meaningless term $M'$ has no normal form of the form
    $\ctorapp{c}{\vec{Q}}{\vec{R}}$, and so if $\SN(M')$ then
    $\pair{M'}{\dummy}$ is in
    $\interpd{\ctorapp{a}{\vec{M}}{\vec{N}}}$.
  \end{description}
\end{myproof}
\fi


%% bottom
% \begin{lemma}
%   \label{lemma:bottom}
%   If $\pair{\sigma(A)}{\interp{A}}\in\interp[]{s}$ then
%   $\pair{\bot}{\interp{\bot}}\in\interp{A}$.
% \end{lemma}

% \begin{myproof}
%   Immediate by \CRfour.
% \end{myproof}


%% variables
\begin{lemma}
  \label{lemma:variables}
  If $\tctypej{\Delta}{\sigma}{\Gamma'}$ and $\interp{x}$ is defined then
  $\triple{\Gamma'}{\sigma(x)}{\interp{x}}\in\interpd{A}$.
\end{lemma}

\iftechreport
\begin{myproof}
  The assumption $\tctypej{\Delta}{\sigma}{\Gamma'}$ implies
  $\triple{\Gamma'}{\sigma(x)}{\interp{x}}\in\interpd[\iargs{\Gamma_1}{\Delta_1}]{A}$
  for some prefixes $\Gamma_1$ and $\Delta_1$ of $\Gamma$ and
  $\Delta$, respectively, and
  $\interpd[\iargs{\Gamma_1}{\Delta_1}]{A}=\interpd{A}$ by
  Interpretation Weakening (Lemma \ref{lemma:weakening}).
\end{myproof}
\fi


%% recursive variables
\begin{lemma}
  \label{lemma:recursive-variables}
  If $\tctypej{\Delta}{\sigma}{\Gamma'}$ for
  $\guform{|\vec{N}|}{\vec{x}}{\pigamma{\Gamma_u}{B}}\in\Gamma$, and
  if $\vec{N},x_i$ are reducible at the types listed in $\Gamma_u$,
  then
  $\pair{\sigma(u)\;\vec{N}\;\sigma(x_i)}{\interp{u\;\vec{N}\;x_i}}\in\interpd{B}$.
\end{lemma}

\iftechreport
\begin{myproof}
  The assumption that $\tctypej{\Delta}{\sigma}{\Gamma'}$ and that the arguments
  $\vec{N},x_i$ are reducible implies that $u\;\vec{N}\;x_i$ is
  reducible relative to $\Delta_1$, as in the proof of Lemma
  \ref{lemma:variables}. The result again follows by Interpretation
  Weakening (Lemma \ref{lemma:weakening}).
\end{myproof}
\fi


%% applications
\begin{lemma}
  \label{lemma:applications}
  If $\pair{\sigma(M)}{\interp{M}}\in\interpd{\pitype{x}{A}{B}}$ and
  $\pair{\sigma(N)}{\interp{N}}\in\interpd{A}$ then
  $\pair{\sigma(M\;N)}{\interp{M\;N}}\in\interpd{[N/x]B}$.
\end{lemma}

\iftechreport
\begin{myproof}
  By hypothesis we immediately have
  $\pair{\sigma(M\;N)}{\interp{M\;N}}\in
  \interpd[\Delta,\dxform{\interp{N}}]{B}$, and the result
  then follows by Interpretation Substitution (Lemma
  \ref{lemma:substitution}).
\end{myproof}
\fi


%% lambda-abstractions
\begin{lemma}
  \label{lemma:abstractions}
  Let $A$ and $B$ be reducible at types $s_A$ and $s_B$, respectively.
  If, for every $\pair{N}{I}\in\interpd{A}$, we have
  $\pair{[\sigma,N/x](B)}{\interp[\Delta,\dxform{I}]{B}}\in\interpd[]{s}$ and
  also $\pair{[\sigma,N/x]M}{\interp[\Delta,\dxform{I}]{M}}\in
  \interpd[\Delta,\dxform{I}]{[\sigma,N/x]B}$, then
  $\pair{\sigma(\lamabs{x}{A}{M})}{\interp{\lamabs{x}{A}{M}}}\in
  \interpd{\pitype{x}{A}{B}}$.
\end{lemma}

\iftechreport
\begin{myproof}
  We must show that, for all $\pair{N}{I}\in\interpd{A}$, we have
  $\pair{\sigma(\lamabs{x}{A}{M})\;N}{\interp[\Delta,\dxform{I}]{M}}
  \in\interp[\Delta,\dxform{I}]{B}$. By the definition of
  $\interpd[]{s_A}$ we have $\SN(A)$, and by \CRfour\ for $A$ we have
  $\pair{\bot}{\dummy}\in\interpd{A}$, so by assumption and by \CRone\
  for $B$ we have $\SN([\sigma,\bot/x]M)$ and thus $\SN(\sigma(M))$.
  It is straightforward to see that $\sigma(\lamabs{x}{A}{M})$ is the
  same as $\lamabs{x}{(\sigma(A))}{(\sigma(M))}$ and that
  $[\sigma,N/x]M$ is the same as $[N/x](\sigma(M))$, and thus we have
  $\sigma(\lamabs{x}{A}{M})\;N\rrtocbnnc [\sigma,N/x]M$. Hence, by
  \CRthree\ for $B$, we have the desired result.
\end{myproof}
\fi


%% constructors
\begin{lemma}
  \label{lemma:constructors}
  Let the type $\ctorapp{a}{\vec{M}}{\vec{Q}}$ and the context
  $\Gammap,\Gammac$ of arguments to constructor $c$ be reducible. If
  $\pair{\sigma(\vec{M},\vec{N})}{\interp{\vec{M},\vec{N}}}\in\interpd{\Gammap,\Gammac}$
  then $\ctorapp{c}{\vec{M}}{\vec{N}}$ is reducible at type
  $\ctorapp{a}{\vec{M}}{\vec{Q}}$.
\end{lemma}

\iftechreport
\begin{myproof}
  By \CRone\ for the types in $\Gammac$, we have that
  $\SN(\sigma(M_i))$ and $\SN(\sigma(N_j))$ for all $i$ and $j$. Thus
  $\SN(\sigma(\ctorapp{c}{\vec{M}}{\vec{N}}))$.  For the third
  condition in the interpretation of $a$, if
  $\sigma(\ctorapp{c}{\vec{M}}{\vec{N}})\rrtostar\ctorapp{c}{\vec{M'}}{\vec{N'}}$
  then we must have $\sigma(M_i)\rrtostar M_i'$ and
  $\sigma(N_j)\rrtostar N_j'$ for all $i$ and $j$. By assumption and
  by \CRtwo\ for the $A_i$ we then have that
  $\pair{(\vec{M'},\vec{N'})}{\interp{\vec{M},\vec{N}}}\in\interpd{\Gammac}$.
  Further, if $a$ is a predicative inductive type then we cannot have
  $c:\Prop$, and we have
  $\interp{\ctorapp{c}{\vec{M}}{\vec{N}}}=\pair{c}{\interp{\vec{N}}}$.
  In contrast, if $a$ is an impredicative inductive type then
  $c:\Prop$, so $\interp{\ctorapp{c}{\vec{M}}{\vec{N}}}=\dummy$.  It
  is not possible for $\NF(\sigma(\ctorapp{c}{\vec{M}}{\vec{N}}))$ to
  not start with $c$.
\end{myproof}
\fi


%% pattern-matching
\begin{lemma}
  \label{lemma:pattern-matching}
  Let
  $\Mfun=\pmfun[ip]{u}{\Gammaarg}{\pc{\vec{c}}{\vec{\Gamma}}{\vec{M}}}$
  have type $\Afun$, and let the context
  $\Gammaarg,y:\ctorapp{a}{\vec{Q}}{\vec{R}},z:B$ be reducible with
  respect to $\Delta$.  If $\tctypej{\Delta}{\sigma}{\Gamma'}$, and if
  \[
  \pair{\sigma_i(M_i)}{\interp[\Delta_i]{M_i}}\in\interpd[\Delta_i,\dxform{\interp[\Delta_i]{\ctorapp{c_i}{\vec{Q}}{\Gamma_i}}}]{B}
  \]
  holds for all $\Delta_i$ and $\sigma_i$ such that
  $\tctypej[\Gamma_i']{\Delta_i}{\sigma_i}{\Gamma'}$, where $\Gamma_i'$ is
  \[
  \Gamma,\Gammaarg,\Gamma_i,\guform{|\Gammaarg|}{\Gamma_i}{\Afun},
  \]
  then $\pair{\sigma(\Mfun)}{\interp{\Mfun}}\in\interpd{A}$.
\end{lemma}

\iftechreport
\begin{myproof}
  For all
  $\pair{(\vec{N},N)}{(\vec{I},I)}\in\interpd{\Gammaarg,x:\ctorapp{a}{\vec{Q}}{\vec{R}}}$,
  we must show that $\pair{M}{I_M}$ is in
  $\interpd[\Delta,\Gammaarg\mapsto\vec{I},\dxform{I}]{B}$ where
  $M\equiv\sigma(\Mfun)\;\vec{N}\;N$ and
  $I_M\equiv\appstar{\interp{\Mfun}}{(\vec{I},I)}$.  By \CRfour\ for
  the types in $\Gammaarg$ and $\Gamma_i$ we can form $\Deltadi$ and
  $\sigmadi$ such that that extend $\Delta$ and $\sigma$,
  respectively, by mapping $u$ and the variables of the contexts
  $\Gammaarg$ and $\Gamma_i$ to $\bot$ in $\sigmadi$ and $\dummy$ in
  $\Deltadi$. Thus we have
  $\pair{\sigmadi(M_i)}{\interp[\Deltadi]{M_i}}\in\interpd[\Deltadi,\dxform{\dummy}]{B}$,
  so by \CRone\ for $B$ we have $\SN(\sigmadi(M_i))$ and hence
  $\SN(\sigma(M_i))$. By \CRone\ for $\Gammaarg$ and
  $\ctorapp{a}{\vec{Q}}{\vec{R}}$ we also have $\SN(\vec{N})$ and
  $\SN(N)$. We now proceed by primary induction on $\NF(N)$ and by
  secondary induction on $\size{N}$.

  For the step case of the secondary induction, if $N\rrtocbnnc N'$
  then $\size{N'}<\size{N}$, so by the inductive hypothesis we have
  that
  $\pair{\sigma(\Mfun)\;\vec{N}\;N'}{\appstar{\interp{\Mfun}}{\pair{(\vec{N},N)}{(\vec{I},I)}}}$
  is in the required set, and the result follows by \CRthree\ for $B$.

  For the base case of the secondary induction, if $N$ is a WHNF then
  it is straightforward to see that it is either of the form
  $\ctorapp{c_i}{\vec{Q}}{\vec{R'}}$ or it does not reduce to a term
  of this form. In the latter case, the definition of
  $\interpd{\Delta}{\ctorapp{a}{\vec{Q}}{\vec{R}}}$ ensures that
  $I=\dummy$, and so the interpretation of $\Mfun$ return
  $I_M=\dummy$.  In this case we also have that
  $\sigma(\Mfun)\;\vec{N}\;N$ is a WHNF, and so is a meaningless
  term. The desired result then follows by \CRfour\ for $B$.

  In the former case, the definition of
  $\interpd[]{\ctorapp{a}{\vec{Q}}{\vec{R}}}$ ensures that
  $\pair{\vec{R'}}{\vec{I'}}\in\interpd{\Gammaci}$ for some
  $\vec{I'}$, since $N\rrtostar N$. Thus we can form $\Delta'$ and
  $\sigma'$ extend $\Delta$ and $\sigma$ by mapping $\Gammaarg$ to
  $\vec{I}$ and $\vec{N}$, respectively, by mapping $\Gamma_i$ to
  $\vec{I'}$ and $\vec{R'}$, respectively, and by mapping $u$ to
  $\sigma(\Mfun)$ and $\interp{\Mfun}$, respectively.  We can conclude
  $\tctypej[\Gamma_i']{\Delta'}{\sigma'}{\Gamma'}$ for $\Gamma_i'$ as above,
  using the primary induction hypothesis and the fact that $\NF(R'_i)$
  is a strict subterm of $\NF(N)$ to handle the mapping in $\Delta'$
  for $u$.  Thus by assumption we have
  $\pair{\sigma'(M_i)}{\interp[\Delta']{M_i}}\in\interpd[\Delta',\dxform{I_x}]{B}$
  where $I_x=\interp[\Delta']{\ctorapp{c_i}{\vec{Q}}{\Gamma_i}}$.

  If we know that $\interp[\Delta']{M_i}=I_M$ then the result follows
  by \CRthree\ for $B$, as $\sigma(M)\rrtocbnnc\sigma'(M_i)$.  For
  predicaitve pattern-matching functions, this is immediate by the
  definition of $\interp{\Mfun}$ and by Interpretation Substitution
  (Lemma \ref{lemma:substitution}) as $I=\pair{c_i}{\vec{I'}}$. For
  impredicative pattern-matching functions with propositional return
  type, we know by Proof Irrelevance (Lemma
  \ref{lemma:proof-irrelevance}) that $\interp[\Delta']{M_j}=\dummy$
  for all $j$, and thus $I_M=\dummy$ as well. Finally, for
  impredicative pattern-matching functions with predicative return
  type, we must have that $a$ has exactly one constructor that takes
  only proof arguments (since there are no constructors $c_i$ for
  empty inductive types). Thus, by Proof Irrelevance, each $I'_j$ has
  only one possible value, $I'_j=\dummy$. This means there is only one
  possible well-formed extension $\Delta'$ of $\Delta$ and we have
  that $I_M$ is well-defined and $I_M=\interp[\Delta']{M_j}$.
\end{myproof}
\fi



%%%%%%%%%%%%%%%%%%%%%%%%%%%%%%%%%%%%%%%%%%%%%%%%%%%%%%%%%%%%%%%%%%%%%%
%%%%% Summarized Proof Details for Conference Paper              %%%%%
%%%%%%%%%%%%%%%%%%%%%%%%%%%%%%%%%%%%%%%%%%%%%%%%%%%%%%%%%%%%%%%%%%%%%%
\else
%
Proof is by the same induction scheme as used to define
$\interp{\cdot}$.  For reasons of space, we consider only a few of the
more interesting cases, at a high level; the remaining cases are
either straightforward, such as subtyping (by Lemma
\ref{lemma:subtyping}), variables $x$, and recursive variables $u$
(the latter two by well-formedness of $\Xi$), and/or are similar to
previous approaches, such as the original proof by Girard
\cite{girard-proofs-types}.
% FIXME: cite tech report?
% See the companion technical report
% \cite{westbrook-techreport} for more details.

If $M=\ctorapp{a}{\vec{M}}{\vec{N}}$, we know $A=s$ for some $s$, and
we must prove $\triple{\Gamma'}{\sigma(M)}{\interp{M}}$ is in
$\interp{s}$, meaning that $\SN(M)$, $\typej[\Gamma']{\sigma(M)}{s}$, and
$\CR{\Gamma}{A}{I}$. The first is immediate by the tertiary IH and
\CRone, the second follows by Substitution, and the third is
straightforward for most of the conditions for reducibility
candidates. For example, \CRtwo\ follows because, if $N\rrto N'$ and
$N'\rrtostar\ctorapp{c}{\vec{Q}}{\vec{R}}$ then
$N\rrtostar\ctorapp{c}{\vec{Q}}{\vec{R}}$. \CRfour\ follows because,
if $N$ is stuck, then $N\rrtostar\ctorapp{c}{\vec{Q}}{\vec{R}}$ is not
possible. The most difficult condition to show is \CRthree, which
requires showing that, if $N'\rrtocbnnc N$ and
$N'\rrtostar\ctorapp{c}{\vec{Q}}{\vec{R}}$ then
$N\rrtostar\ctorapp{c}{\vec{Q}}{\vec{R}}$. Note that this is a much
weaker property than Confluence. It follows because any reduction to a
constructor application must eventually perform a CBN reduction step,
and any non-CBN reduction step can always be commuted (to zero or more
steps) after a CBN step.
% if $\triple{\Gamma''}{N}{I}\in\interpd{M}$ and
% $N'\rrtocbnnc N$ then $\triple{\Gamma''}{N'}{I}\in\interpd{M}$. This
% requires showing that $N'\rrtostar\ctorapp{c}{\vec{Q}}{\vec{R}}$
% implies $N\rrtostar\ctorapp{c}{\vec{Q}}{\vec{R}}$, which
% holds because any sequence of reductions from $N'$ to
% a constructor application must eventually reduce the top-level
% redex that causes $N'$ to reduce to $N$, and any earlier
% reductions can be commuted after this key reduction, yielding a
% reduction from $N$.


The other case we consider here is if $M$ is a pattern-matching
function with pattern cases $M_i$ over type
$\ctorapp{a}{\vec{N}}{\vec{Q}}$ with return type $B$.  Since the
assumption $\typej{\Afun}{s}$ is part of the typing rule for $M$, we
immediately have, by the tertiary IH, that the interpretations of $B$
and of all the argument types are reducibility candidates.  If $a$ is
impredicative, we first must show that $\interp{M}$ is well-defined,
meaning all $M_i$ have the same interpretations for all valid
extensions $\Xi,\dgform[\Gamma_i]{\vec{N}}{\vec{I}}$ of $\Xi$.  This follows
by Proof Irrelevance, because either there is at most one pattern case
$M_i$ and at most one possible sequence of $\vec{I}$, namely
$I_j=\dummy$ for all $j$, or all possible interpretations of the $M_i$
equal $\dummy$. We then must show that
$\triple{\Delta'}{\Dsubst(M)\;\vec{R}\;R_0}{\appstar{\appstar{\interp{M}}{\vec{I}}}{I_0}}$
is in $\interp{B}$ whenever $\triple{\Delta'}{R_j}{I_j}$ is in the
interpretation of the appropriate argument type. This is shown by
\CRthree, using induction on the number of CBN steps it takes to
either reduce $R_0$ to a constructor application or a stuck term.
This induction is possible since $\SN(R_0)$ follows by \CRone\ for
$\ctorapp{a}{\vec{N}}{\vec{Q}}$.
%
\fi
\end{myproof}


\begin{corollary}(Strong Normalization)
If $\typej{M}{A}$ then $\SN(M)$.
\end{corollary}

% \begin{myproof}
%   By Lemma \ref{lemma:type-typing}, $\typej{A}{s}$ for some $s$, and
%   since (as discussed above)
%   $\ivtypej{\Gamma\mapsto\dummy}{\Gamma}$, Reducibility implies
%   that
%   $\triple{\Gamma}{A}{\interp[\Gamma\mapsto\dummy]{A}}\in\interpd[]{s}$.
%   Thus, again by Reducibility,
%   $\triple{\Gamma}{M}{\interp[\Gamma\mapsto\dummy]{M}}\in\interpd[]{A}$,
%   and so, by \CRone, $\SN(M)$.
% \end{myproof}


%%%%%%%%%%%%%%%%%%%
%%%%% Section %%%%%
%%%%%%%%%%%%%%%%%%%
% \section{The Calculus of Nominal Inductive Constructions (CNIC)}
% \label{sec:cnic}



%%%%%%%%%%%%%%%%%%%
%%%%% Section %%%%%
%%%%%%%%%%%%%%%%%%%
% \section{Strong Normalization for CNIC}
% \label{sec:cnic-sn}



%%%%%%%%%%%%%%%%%%%
%%%%% Section %%%%%
%%%%%%%%%%%%%%%%%%%
\section{Related Work}
\label{sec:related-work}

There have been numerous proofs of SN for various forms of both the
Calculus of Inductive Constructions (CiC) and the Calculus of
Constructions without inductive types (CC). As discussed above, none
has been extended to full CiC. Many of these proofs are based on the
\emph{Candidates de Reducibilit\'{e}} of Girard
\cite{girard-proofs-types}, or on a modified version called
\emph{saturated sets}.  Under these approaches, each type is
interpreted as a set of terms that satisfies some closure properties
(such as properties \CRone\ through \CRfour\ above), which include
strong normalization of the elements. The proof then proves the
Reducibility Theorem, which states that each term is in the
interpretation of its type, and SN follows.

Using this approach, Luo \cite{luo90} proves SN for the Extended
Calculus of Constructions (ECC), a version of CC with an impredicative
universe ($\Prop$), a hierarchy of infinitely many predicative
universes ($\Type_i$), and product types. This seems to be the
only SN proof for a calculus with infinitely many universes, probably
due to the non-uniformity problem discussed in the Introduction.
Luo's proof defines an interpretation for each type construct, and
then proves Reducibility. The interpretation only defined for
types that are in weak head normal form, however. Thus Luo
requires a complex and involved Quasi-Normalization theorem, which
states that all terms can be reduced to a weak head normal form.

Coquand and Gallier \cite{coquand90} adapt the saturated sets approach
to work on typed sets of terms, yielding a proof of SN for CC with one
universe. They call their interpretation a Kripke structure, as each
type is interpreted with respect to a particular world. The
interpretation of a type contains pairs $\pair{\Delta'}{M}$ of
interpretation contexts $\Delta'$ that extend $\Delta$ and terms $M$
such that $M$ is well-typed with respect to (the context associated
with) $\Delta'$. This was the inspiration for including contexts
$\Gamma$ in our triples $\triple{\Gamma}{M}{I}$.  Again, because of
non-uniformity, however, it is unclear how to extend this approach to
multiple universes. It is also interesting to consider how sets of
typed terms would make our approach more difficult.

Geuvers \cite{geuvers94} extends the saturated sets approach by
defining two interpretations, one for terms and one for types.  The
interpretation for terms maps a term to a substitution instance of
that term, and Reducibility then requires only that the instance be in
the interpretation of a related substitution instance of the type of
the term. Geuvers then shows that this approach can be used to support
CC with strong products and inductive kinds, but comments that it will
probably not work with small inductive types like the natural numbers.

Altenkirch \cite{altenkirch93} adapts the notion of saturated sets to
be a semantic notion, using the saturated sets as the basis for a
denotational idea called a \emph{CC-structure} and thus building a
model of CC. Soundness of this model then implies SN of CC. Altenkirch
then shows that this argument can be extended to handle inductive
types.  In a similar manner, Goguen \cite{goguen94} proves SN by
providing a typed operational semantics of terminating computations of
a given type. He then shows that this semantics is sound for the
syntactic type theory, i.e., that all well-typed terms have a
corresponding terminating computation. This result is proved for CiC
with with two universes, one predicative and one impredicative.

Finally, Geuvers and Nederhof \cite{geuvers91} prove SN for CC without
universes by mapping terms in this language into System
$\mathrm{F}_{\omega}$. The proof for SN of System
$\mathrm{F}_{\omega}$ uses saturated sets but is much simpler, because
types are trivially strongly normalizing. It is not clear, however,
that this approach could be adapted easily (without proving SN
already) to extensions of CC.



%%%%%%%%%%%%%%%%%%%
%%%%% Section %%%%%
%%%%%%%%%%%%%%%%%%%
\section{Conclusion}
\label{sec:conclusion}

In this work, we have shown how to adapt the Logical Relations proof
technique to prove strong normalization (SN) for the full Calculus of
Inductive Constructions (CiC) with a cumulative predicative type
hierarchy. The guiding principle that has allowed us to handle full
CiC is \emph{uniformity}, meaning that our interpretation does not
depend on whether a term is a type. One benefit of this is that our
interpretation does not depend on confluence.  The key technical
development that enables uniformity is to interpret types not as sets
of terms but as sets of triples $\triple{\Gamma}{M}{I}$ of context $\Gamma$,
terms $M$ that are well-typed under $\Gamma$, and valid
interpretations $I$ of $M$. Closure conditions on the sets associated
with types, for example that the sets be saturated sets, are handled
simply by including only such sets as valid interpretations of types
in the interpretations $\interpnod{\Prop}$ and $\interpnod{\Type_i}$ of
universes. Surprisingly, the proof is actually very similar to the
original proof given by Girard, with the proof of the closure
conditions on interpretations of types being folded into the proof of
Reducibility, leading to a simple and elegant proof.  In fact, the
proof given by Girard can be seen as a special case of Uniform Logical
Relations.


As a final consideration, the interpretation we give can easily be
extended to support a transfinite hierarchy of universes.  That is, we
could support universes $\Type_{\alpha}$ for any ordinal $\alpha$. Of
course, to define such a theory syntactically requires a notation of
well-founded ordinals, such as the well-known notation based on
$\phi$-functions for ordinals less than $\Gamma_0$ (see
e.g.~\cite{gallier91}). Thus not only is Zermelo-Frankel set theory
(ZF) proof-theoretically stronger than CiC --- since we have proved
consistency of CiC in ZF --- but, if we view CiC as an
ordinal operation that assigns to each $\alpha$ the proof-theoretic
ordinal of CiC with up to $\alpha$ universes, then the ordinal of ZF
is also larger then this function applied to $\alpha$ (as well as
infinitely many fixed points of this function).
%\footnote{ZF is known to have the same proof-theoretic strength as
%  ZF.}
See e.g.\ Coquand
et al.\ \cite{coquand97} for more on proof-theoretic ordinals of type
theory.

It has been shown by Aczel \cite{aczel77}, however, that
\emph{constructive} ZF (CZF), without the Axiom of the Excluded
Middle, can be encoded in CiC, and so the proof-theoretic strength of
CZF is no more than that of CiC. Note that our intperpretation
critically relies on Excluded Middle to define the interpretations
of inductive types. If we add excluded middle for
each sort $s$ ($EM_s$) to CiC, then the work of Werner \cite{werner97}
shows that full ZF can be encoded, where excluded middle for sort $s$ has
the following type:\footnote{Werner actually uses $EM_{\Prop}$ along
  with the Type-Theoretical Description Axioms at each sort $s$
  ($TTDA_s$), but the author has verified in the Coq proof assistant
  that $EM_s$ implies $TTDA_s$.}
\[
\pitype{A}{s}{A +_s (A \to \mathsf{False})}
\]
Here $\mathsf{False}$ is the empty inductive type and $+_i$ denotes
the inductive type for disjunction in sort $s$.  In fact, Werner shows
that ZF with $i$ inaccessible cardinals ($\ZFi$) can be encoded in CiC
with $EM_{\Type_{i+2}}$, where an \emph{inaccessible cardinal} in ZF
is essentially a set that is so big that it includes a whole model of
ZF (and thus \ZFi[i+1] is guaranteed to be proof-theoretically
stronger than \ZFi).  Note that, although it is possible to interpret
$EM_{\Prop}$ as $\dummy$ in our construction (because of Proof
Irrelevance), it is not possible to interpret $EM_{\Type_i}$ since,
for some types $A$, neither $A$ nor its negation is inhabited.  This
is good, because, by the above and by G\"{o}del's Second
Incompleteness Theorem, defining an interpretation that did allow this
in ZF would only be possible if ZF is inconsistent.  In this light, we can
see that (predicative) excluded middle is in fact a very powerful
axiom.


% use section* for acknowledgement
\section*{Acknowledgments}
The author would like to thank Andreas Abel for spotting an important
bug, Chris Casinghino and Peter Hancock for helpful discussions, and
the past anonymous reviewers who caught a number of bugs and omissions
in previous versions of this paper.


% bibliography
\bibliographystyle{plain}
\bibliography{bib}

% that's all folks
\end{document}


%%% Local Variables: 
%%% mode: latex
%%% End: 
