
\documentclass[conference]{IEEEtran}

\usepackage[cmex10]{amsmath}
\usepackage{amssymb}
\usepackage{stmaryrd}
\usepackage{proof}

\begin{document}

%%% my commands %%%


%%%
%%% spacing
%%%

\newlength{\indlen}
\setlength{\indlen}{8pt}
\newcommand{\ind}[1]{\hspace{#1\indlen}} % stands for indent
\newcommand{\figfill}[1][9pt]{\hspace{#1}\hfill}


%%%
%%% theorem/proof constructs
%%%

\newtheorem{theorem}{Theorem}
\newtheorem{lemma}{Lemma}
\newtheorem{definition}{Definition}
\newtheorem{corollary}{Corollary}

% proof environment
\iftechreport
\newenvironment{myproof}
               {\hspace{-1\parindent}\textbf{Proof:}}
               {\hfill $\square$}
\else
\newenvironment{myproof}
               {\begin{IEEEproof}}
               {\end{IEEEproof}}
\fi


%%%
%%% for writing grammars
%%%

\newcommand{\borspacing}{\hspace{.5em}}
\newcommand{\doublebar}{\protect\rule[-2pt]{1pt}{1em}\hspace{1pt}\protect\rule[-2pt]{1pt}{1em}}
\newcommand{\singlebar}{\protect\rule[-2pt]{1pt}{1em}}
\newcommand{\borstar}{\borspacing\singlebar}
%\newcommand{\borstar}{\doublebar\borspacing}
%\newcommand{\borstar}{|\borspacing}
\newcommand{\bor}{\borstar\borspacing} % stands for bar-or, for use in grammars


%%%
%%% for repetitive commands that just set the font face
%%%

\newcommand{\definermcommand}[2]{\newcommand{#1}{\ensuremath{\mathrm{#2}}}}
\newcommand{\definesfcommand}[2]{\newcommand{#1}{\ensuremath{\mathsf{#2}}}}
\newcommand{\definebfcommand}[2]{\newcommand{#1}{\ensuremath{\mathbf{#2}}}}


%%%
%%% type theory constructs
%%%

\definesfcommand{\Type}{Type}
\definesfcommand{\Prop}{Prop}
\definesfcommand{\Name}{Name}

\newcommand{\To}{\Rightarrow}
\newcommand{\lamabs}[3]{\lambda\,#1 \! : \! #2 \,.\, #3}
\newcommand{\lamabsnot}[2]{\lambda #1.\,#2}
\newcommand{\tlamabsnot}[2]{\Lambda #1.\,#2}
\newcommand{\lamgamma}[2][\Gamma]{\lambda #1\,.\,#2}
\newcommand{\nuabs}[3]{\nu\,#1 \! : \! #2 \,.\, #3}
\newcommand{\nugamma}[2]{\nu\,#1 \,.\, #2}
\newcommand{\nuabsnot}[2]{\nu\,#1\,.\,#2}
\newcommand{\pitype}[3]{\Pi #1\!:\!#2\,.\,#3}
\newcommand{\pitypenot}[2]{\Pi #1\,.\,#2}
\newcommand{\pigamma}[2]{\Pi #1\,.\,#2}
\newcommand{\nabtype}[3]{\nabla #1\!:\!#2\,.\,#3}
\newcommand{\nabgamma}[2]{\nabla #1\,.\,#2}
\newcommand{\namearg}[1]{\ensuremath{@\;#1}}
\newcommand{\namerepl}[2]{\ensuremath{#1\;\namearg{#2}}}
\newcommand{\subtype}{\ensuremath{<:}}
\newcommand{\ctorapp}[3]{#1(#2; #3)}
\newcommand{\uapp}[2]{#1(#2)}

%% sorts
\definermcommand{\impred}{i}
\definermcommand{\pred}{p}
\newcommand{\spred}{s^{\pred}}
\newcommand{\simpred}{s^{\impred}}
\newcommand{\sip}{s^{ip}}

%% pattern-matching functions
\newcommand{\funmatch}[5]{\mathbf{fun}\;#1\;(#2)\;(#3)\;(#4;#5)}
\newcommand{\funmatchi}[6]{\mathbf{fun}^{#1}\;#2\;(#3)\;(#4)\;(#5;#6)}
\newcommand{\pattern}[2]{#1 \backslash #2}
\newcommand{\pc}[3]{\pcnoctxt{\pattern{#1}{#2}}{#3}}
\newcommand{\pcnoctxt}[2]{#1 \to #2}
\newcommand{\pmfun}[4][]{\mathbf{fun}^{#1}\;#2\;(#3)\;(#4)}
\newcommand{\ppmfun}[3]{\pmfun[\pred]{#1}{#2}{#3}}
\newcommand{\ipmfun}[3]{\pmfun[\impred]{#1}{#2}{#3}}
\newcommand{\pmfunsimp}[1]{\mathbf{fun}\;(#1)}
\newcommand{\caseof}[2]{\mathbf{case}\;#1\;\mathbf{of}\;#2}

%% contexts
\newcommand{\gxform}[2][x]{#1:#2}
\newcommand{\guform}[4][u]{#1:^{#2;#3} #4}
\newcommand{\Gammaarg}{\Gamma_{\mathsf{arg}}}
\newcommand{\Gammap}{\Gamma_{\mathsf{p}}}
\newcommand{\Gammaa}{\Gamma_{\mathsf{a}}}
\newcommand{\Gammac}{\Gamma_{\mathsf{c}}}
\newcommand{\Gammaci}[1][i]{\Gamma_{\mathsf{c}_{#1}}}
\newcommand{\ctxtsel}[2]{#1|_{#2}}


%%%
%%% examples
%%%

\definesfcommand{\nat}{nat}
\definesfcommand{\z}{z}
\definesfcommand{\s}{s}
\definesfcommand{\False}{False}

\definesfcommand{\ord}{ord}
\definesfcommand{\ordz}{zero}
\definesfcommand{\ords}{succ}
\definesfcommand{\ordlim}{lim}


%%%
%%% static and operational semantics
%%%

\newcommand{\ctortp}[2]{(#1;\; #2)}
\newcommand{\typej}[3][\Gamma]{#1 \vdash #2 : #3}
\newcommand{\validj}[2][\Gamma]{#1 \vdash #2}
\newcommand{\rrto}{\longrightarrow}
\newcommand{\rrtostar}{\longrightarrow^{*}}
\newcommand{\conv}{\longleftrightarrow^{*}}
\newcommand{\convone}{\longleftrightarrow}
\newcommand{\uform}[4]{#1:^{#2;#3} #4}

%% special forms of reduction
\newcommand{\rrtocbn}{\rrto_{\mathrm{n}}}
\newcommand{\rrtocbnnc}{\rrto_{\mathrm{n-nc}}}

%% typing for contexts
\newcommand{\wfctxtj}[1]{\vdash #1}
\newcommand{\wfj}[2][\Gamma]{#1\vdash #2}


%%%
%%% helpers for interpretation function
%%%

%% ordered tuples
\newcommand{\onetuple}[1]{\langle #1\rangle}
\newcommand{\pair}[2]{\langle #1 ,\; #2\rangle}
\newcommand{\triple}[3]{\langle #1 ,\; #2, \; #3\rangle}

%% my own special logical operators, with extra spacing and text where needed
\newcommand{\mywedge}{\,\wedge\,}
\newcommand{\myvee}{\,\vee\,}
\newcommand{\myimplies}{\text{ \textbf{implies} }}

%% set comprehension
\newcommand{\setcompr}[2]{\{\, #1 \;|\;\; #2 \,\}}
\newcommand{\setcomprarr}[2]{\{\, #1 \;|\;\; \begin{array}[t]{@{}l@{}} #2 \,\}\end{array}}
\newcommand{\setcomprlong}[2]{\begin{array}[t]{@{}l@{}}\{\, #1 \\\;\;\;\;|\;\; \begin{array}[t]{@{}l@{}}#2 \,\}\end{array}\end{array}}

%% if-then-else
\newcommand{\ifthen}[3]{\text{\textbf{if} }#1\text{ \textbf{then} }#2\text{ \textbf{else} }#3}
\newcommand{\ifthennl}[3]{\text{\textbf{if} }#1\begin{array}[t]{@{}l@{}}\text{ \textbf{then} }#2\\\text{ \textbf{else} }#3\end{array}}

%% primitive recursion
\newcommand{\primrecname}[1][\dummy]{\mathbf{primrec}^{#1}}
\newcommand{\primrec}[4][\dummy]{\primrecname[#1]_{#2}\;(#3)\;(#4)}
\newcommand{\primrecarr}[4][\dummy]{\primrecname[#1]_{#2}\;(#3)\;(\begin{array}[t]{@{}l@{}}#4)\end{array}}
\newcommand{\primrecnl}[4][\dummy]{\primrecname[#1]_{#2}\;\begin{array}[t]{@{}l@{}}(#3)\\(#4)\end{array}}
% \newcommand{\primrecname}{\mathbf{primrec}}
% \newcommand{\primrec}[3]{\primrecname_{#1}\;(#2)\;(#3)}

%% "dummy" definitions
\newcommand{\lamstar}[3]{\lambda^{\dummy} (#1\in #2) .\; #3}
\newcommand{\lamstarnot}[2]{\lambda^{\dummy} #1 .\; #2}
\newcommand{\pistar}[2]{#1 \to^{\dummy} #2}
\newcommand{\appstar}[2]{#1 \;@^{\dummy}\; #2}
\newcommand{\ctorstar}[2]{#1^{\dummy}(#2)}
%\newcommand{\instar}{\in^{\dummy}}
\newcommand{\Deltadi}[1][i]{\Delta_{\dummy,#1}}
\newcommand{\sigmadi}[1][i]{\sigma_{\dummy,#1}}

%% forming Deltas and sigmas
\newcommand{\dxform}[3][x]{#1\mapsto (#2,#3)}
%\newcommand{\duform}[4][u]{#1:(#2;#3)\mapsto #4}
\newcommand{\duform}[3][u]{#1\mapsto (#2,#3)}
\newcommand{\dgform}[3][\Gamma]{#1\mapsto (#2,#3)}
\newcommand{\dgargform}[3]{#1\mapsto (#2,#3)}

%% well-formedness
%\newcommand{\wfdsj}[3][\Gamma]{#1 \vdash (#2;#3)}
%\newcommand{\wfdsjdef}{\wfdsj[\Gamma]{\Delta}{\sigma}}
\newcommand{\tctypej}[4][\Gamma]{#1 \vdash (#2 ; #3) : #4}
\newcommand{\tctypejdef}[1][\Gamma']{\tctypej{\Delta}{\sigma}{#1}}
\newcommand{\typejd}[2]{\typej[\dctxt{\Delta}]{#1}{#2}}
\newcommand{\typejemp}[2]{\typej[\cdot]{#1}{#2}}
\newcommand{\typejdg}[2]{\typej[\dctxt{\Delta},\Gamma]{#1}{#2}}
\newcommand{\validjd}[1]{\validj[\dctxt{\Delta}]{#1}}
\newcommand{\ivtypej}[3][\Gamma]{#1 \vdash #2 : #3}

%% misc
\newcommand{\Iarg}{I_{\mathsf{arg}}}
\newcommand{\size}[1]{\mathbf{size}(#1)}
\newcommand{\Mfun}{\ensuremath{M_{\mathrm{fun}}}}
\newcommand{\Afun}{\ensuremath{A_{\mathrm{fun}}}}
\newcommand{\proji}[1]{\pi_3 (#1)}
\newcommand{\proj}[2][c]{\mathsf{proj}_{#1,#2}}
\newcommand{\projvec}[1][c]{\vec{\mathsf{proj}_{#1}}}
\newcommand{\F}{\ensuremath{\mathfrak{F}}}
\newcommand{\Fomega}{\ensuremath{\F^{\omega}}}
\newcommand{\Ii}[1][i]{\ensuremath{\mathbb{I}_{#1}}}
\newcommand{\Sii}[2][i]{\ensuremath{\mathbb{S}_{#1,#2}}}
\newcommand{\Ui}[1][i]{\ensuremath{\mathbb{U}_{#1}}}
%\newcommand{\Ii}[1][i]{\ensuremath{\mathbb{I}_{#1}}}
%\newcommand{\Iomega}{\Ii[\omega]}
\newcommand{\liftset}[2]{\ensuremath{\mathcal{U}}_{#1}(#2)}
\newcommand{\powerset}{\ensuremath{\mathcal{P}}}
\definesfcommand{\Ctxt}{Ctxt}
\definesfcommand{\Term}{Term}
\newcommand{\interphole}[2][I]{\ensuremath{#1[#2]}}
\newcommand{\ctors}[1]{\mathrm{ctors}_{\Sigma}(#1)}
\newcommand{\elimok}[2]{\mathbf{elim\text{-}ok}(#1\mapsto #2)}
\definermcommand{\Dom}{Dom}
\definermcommand{\Ran}{Ran}
\newcommand{\countablefun}[2]{#1 \stackrel{\aleph_0}{\to} #2}
\definermcommand{\TC}{TC}
\newcommand{\instar}{\in^{*}}
\definesfcommand{\id}{id}

%% denotations
\newcommand{\denotationnot}[1]{\lceil #1 \rceil}
\newcommand{\denotation}[3]{\lceil #3 \rceil_{#1\vdash #2}}
\newcommand{\denotationd}[1]{\denotation[\dctxt{\Delta}]{#1}}
%\newcommand{\interpd}[2][\Delta]{\denotationd[\dctxt{#1},#2]{\interp[#1]{#2}}}
\newcommand{\interpd}[2][\iargs{\Delta}{\Xi}]{\llbrace #2 \rrbrace^{#1}}
%\newcommand{\interpdS}[2][\Delta]{\llbrace #2 \rrbrace^{#1}_{S}}
%\newcommand{\interpdI}[2][\Delta]{\llbrace #2 \rrbrace^{#1}_{I}}
\newcommand{\llbrace}{\{\hspace{-2.5pt}|}
\newcommand{\rrbrace}{|\hspace{-2.5pt}\}}

%%%
%%% interpretation and related constructs
%%%

%\newcommand{\interp}[2][\Delta]{\llbracket #2 \rrbracket^{#1}}
\newcommand{\iargs}[2]{#1;#2}
\newcommand{\interp}[2][\iargs{\Delta}{\Xi}]{\llbracket #2 \rrbracket^{#1}}
\newcommand{\interpnod}[1]{\interp[]{#1}}
\newcommand{\interpind}[1]{\llbracket #1 \rrbracket_{\mathbf{ind}}}
%\newcommand{\interpSarg}[3]{\llbracket #3 \rrbracket^{#2}_{#1}}
%\newcommand{\interpS}[2][\Delta]{\interpSarg{S}{#1}{#2}}
%\newcommand{\interpI}[2][\Delta]{\interpSarg{I}{#1}{#2}}
\newcommand{\dummy}{\ensuremath{\bullet}}
\newcommand{\dctxt}[1]{|#1|}
\newcommand{\Dsubst}[1][\Xi]{#1^{\sigma}}
\newcommand{\Dlookup}[2][\Xi]{#1^{\mathrm{I}}(#2)}
\definesfcommand{\SN}{SN}
\definesfcommand{\NF}{NF}
\newcommand{\CRnot}[1]{\ensuremath{\mathbf{CR}(#1)}}
\newcommand{\CR}[3]{\ensuremath{\mathbf{CR}_{#1\vdash#2}(#3)}}
\newcommand{\CRone}{\textbf{CR~1}}
\newcommand{\CRtwo}{\textbf{CR~2}}
\newcommand{\CRthree}{\textbf{CR~3}}
\newcommand{\CRfour}{\textbf{CR~4}}
\newcommand{\CRG}{\textbf{G}}
\newcommand{\CRT}{\textbf{T}}
\newcommand{\CRE}{\textbf{E}}


%%%
%%% Conclusion stuff about excluded middle
%%%

\newcommand{\ZFi}[1][i]{\ensuremath{\text{ZF}_{#1}}}
\newcommand{\ZFCi}[1][i]{\ensuremath{\text{ZFC}_{#1}}}

%%% Local Variables: 
%%% mode: latex
%%% TeX-master: "uniform-lr"
%%% End: 



\title{Uniform Logical Relations}
\author{\IEEEauthorblockN{Edwin Westbrook}
\IEEEauthorblockA{Department of Computer Science\\
Rice University\\
Houston, TX 77005\\
Email: emw4@rice.edu}}


\maketitle

\begin{abstract}
The abstract goes here.
\end{abstract}

%\IEEEpeerreviewmaketitle


%%%%%%%%%%%%%%%%%%%
%%%%% Section %%%%%
%%%%%%%%%%%%%%%%%%%
\section{Introduction}

Intensional constructive type theory (ICTT), as first introduced by
Martin-L\"{o}f \cite{martinlof72}, is compelling because
- it gives a combined programming language / mathematical
  theory (see e.g.~\cite{nps90})
- further, computation is defined entirely syntactically; no need
  to appeal to models or set theory
- an important consideration for such theories is SN
  + ensures consistency
  + proof-checking is decidable
  + gives a simple model, i.e., the term algebra
    - easy to construct
    - foundationally compelling because it does not appeal to set theory


Unfortunately, it is not known how to prove SN for more powerful
extensions of ICTT such as the Calculus of Inductive Constructions
(CiC), the basis of the Coq theorem prover \cite{coq}.
- set theoretic models have been given for CiC [FIXME], but these do
  not imply SN only consistency
- the problem is higher universes + large / impredicative inductive types
  + SN for higher universes in predicative theories is well-known [FIXME]
- for impredicative theories we need logical relations
  \cite{girard-proofs-types}
  + types are interpreted as logically defined sets of terms
  + FIXME
- logical relations works for theories with large types, such as
  the Calculus of Constructions \cite{coquand88}; however, it does
  not work for with higher universes
  + the problem: it is non-uniform
  + more specifically, interpreting functions depends on whether the argument
    type is ``\Type''
  + with higher universes this depends on evaluation
FIXME: problem is also that we need quasi-normalization to induct on types
in order to define their LRs

to solve this problem, we introduce a new approach to LR that is uniform
- explain it
- we also show that uniform logical relations can be adapted for non-trivial
  extensions of CiC, such as CNIC \cite{westbrook09,westbrook-thesis}
- in order to make our definition inductive, we have to make all proofs
  be interpreted as a ``dummy'' object $\dummy$
  + this works because proof objects cannot contribute any ``information''
    to the interpretation of a type
  + justifies proof-irrelevance [FIXME]


The remainder of this document is organized as follows. Section
\ref{sec:cic} reviews the syntax and semantics of the Calculus of
Inductive Constructions (CiC), while Section \ref{sec:cic-sn} shows
how to prove Strong Normalization for CiC using Uniform Logical
Relations. Section \ref{sec:cnic} describes the Calculus of Nominal
Inductive Constructions (CNIC), an extension of CiC with support for
nominal features and higher-order encodings.  Section
\ref{sec:cnic-sn} then shows how Uniform Logical Relations can be
adapted for CNIC. Finally, Section \ref{sec:conclusion} concludes.



%%%%%%%%%%%%%%%%%%%
%%%%% Section %%%%%
%%%%%%%%%%%%%%%%%%%
\section{The Calculus of Inductive Constructions (CiC)}
\label{sec:cic}


%%%%%%%%%%%%%%%%%%%
%%%%% Section %%%%%
%%%%%%%%%%%%%%%%%%%
\section{Strong Normalization for CiC}
\label{sec:cic-sn}

the basic idea is to have an interpretation operation $\interpnod{\cdot}$
\begin{itemize}
\item the logical relation for a type $A$ is given by $\interpnod{A}$
\item explain the two roles of the dummy object $\dummy$
  \begin{itemize}
  \item in order for inductiveness to work, we need a ``dummy'' object as
    the interpretation of propositions in \Prop\ (otherwise the interp of
    \Prop\ depends on itself!)
  \item we also need uninterpreted variables; solves one of the issues
    in Girard's original proof (see [FIXME]) with empty domain types of pi
  \item explain $\lamstar{x}{S}{I}$ and $\appstar{F}{I}$
  \end{itemize}
\end{itemize}


\begin{definition}[Denotation]
  We define $\denotation{A}{I}$, the \emph{denotation of $I$ at type
    $A$ and context $\Gamma$}, as follows. If $I=\dummy$ then
  $\denotation{A}{I}$ is:
  \[
  \setcompr{\triple{\Gamma'}{M}{\dummy}}{\typejd{(\Gamma',M)}{A} \mywedge \SN(M)}
  \]
  Otherwise, $\denotation{A}{I}=I$.
\end{definition}

\begin{definition}[Reducibility Candidate]
  We say that $S$ is a \emph{reducibility candidate for $A$ with respect
    to $\Gamma$}, written $\CR{A}{S}$, iff $S$ is a set of triples
  $\triple{\Gamma'}{M}{I}$ such that $\typej{(\Gamma',M)}{A}$ and:
  \begin{description}
  \item[\textbf{(CR 1):}] \hspace*{10pt} $\SN(M)$;
  \item[\textbf{(CR 2):}] \hspace*{10pt} If $M\rrto M'$ then $\pair{\Gamma'}{M'}{I}\in S$; and
  \item[\textbf{(CR 3):}] \hspace*{10pt} If $M'$ is neutral and if
    $\triple{\Gamma'}{M''}{I'}\in S$ for all $M''$ such that $M'\rrto M''$, then
    $\triple{\Gamma'}{M'}{I'}\in S$.
  \end{description}
  where the \emph{neutral} terms are those of the form $x$ and $M\;N$.
\end{definition}


\begin{lemma}
\label{lemma:cr-sn}
Let $S$ be any set of triples $\triple{\Gamma'}{M}{I}$, and define
\[
S' = \setcompr{\pair{\Gamma'}{M}}{\exists I.\triple{\Gamma'}{M}{I}\in S}.
\]
Further, let $S$ satisfy $\triple{\Gamma'}{M_1}{I}\in S$ iff
$\triple{\Gamma'}{M_2}{I}\in S$ whenever $M_1\rrto M_2$ and
$\pair{\Gamma'}{M_i}\in S'$.  If $S'$ is the set of all
$\pair{\Gamma'}{M}$ such that $\typej{(\Gamma',M)}{A}$ and $\SN(M)$,
then $I$ is a reducibility candidate for $A$ with respect to $\Gamma$.
\end{lemma}

\begin{proof}
\CRone\ is immediate. \CRtwo\ follows because $M\rrto M'$ for $\SN(M)$
implies $\SN(M')$. Finally, \CRthree\ follows because, for any $M'$
(including the neutral $M'$), $\SN(M'')$ for all $M''$ such that $M'\rrto M''$
implies $\SN(M')$.
\end{proof}

\begin{corollary}
$\CR{A}{\denotation{A}{\dummy}}$ for all $\Gamma$ and $A$.
\end{corollary}


explain delta:
\[
\Delta ::= \cdot \bor \Delta,x:A \bor \Delta,x:A\mapsto\pair{M}{I}
\]
- explain uninterpreted variables

We define $\interp{\cdot}$ in Figure \ref{fig:cic-lr}.
- we use the notation $\interpd{A}=\denotationd{\interp{A}}$

\begin{figure*}
\centering
\begin{math}
\begin{array}{@{}l@{\hspace{7pt}}c@{\hspace{8pt}}l@{}}
% sorts
\interp{s} & = &
\setcompr{\triple{\Gamma}{A}{I}}{\typejd{(\Gamma,A)}{s} \mywedge
  \SN(A) \mywedge \CR[\dctxt{\Delta},\Gamma]{A}{\denotation[\dctxt{\Delta},\Gamma]{A}{I}}}\\

% Pi-types
\interp{\pitype{x}{A}{B}} & = &
\setcomprlong{\triple{\Gamma}{M}{F}}{
  \typejd{(\Gamma,M)}{(\pitype{x}{A}{B})} \mywedge\\
  \forall(\triple{\Gamma'}{N}{I}\in\interpd[\Delta,\Gamma]{A}) .\\
  \ind{1}\triple{\cdot}{M\;N}{\appstar{F}{\triple{\Gamma'}{N}{I}}} \in\interpd[\Delta,\Gamma,\Gamma',x:A\mapsto \pair{N}{I}]{B}
}
\\[30pt]

% inductive types
\interp{\ctorapp{a^{-1}}{\vec{M}}{\vec{N}}} & = &
\setcompr{\triple{\Gamma}{M}{\dummy}}{
  \typejd{(\Gamma,M)}{\ctorapp{a^{-1}}{\vec{M}}{\vec{N}}} \mywedge \SN(M)
}
\\

\interp{\ctorapp{a^{i}}{\vec{M}}{\vec{N}}} & = &
\setcomprlong{\triple{\Gamma}{M}{\pair{c}{\vec{I}}}}{
  \typejd{(\Gamma,M)}{\ctorapp{a^{i}}{\vec{M}}{\vec{N}}}
  \mywedge \SN(M) \mywedge 
  \NF(M)=\ctorapp{c}{\vec{v'}}{\vec{v}} \mywedge\\
  c : \pigamma{\Gammap}{\pigamma{\Gammaarg}{\ctorapp{a^i}{\Gammap}{\vec{N'}}}} \mywedge
  \pair{\vec{v}}{\vec{I}}\in \interpd[\Delta,\Gammap\mapsto \interp{\vec{M}}]{\Gammaarg}
}
% \interp{\ctorapp{a^{i}}{\vec{M}}{\vec{N}}} & = &
% \setcomprlong{\pair{M}{\pair{c}{\vec{I}}}}{
%   \typej{M}{\ctorapp{a^{i}}{\vec{M}}{\vec{N}}} \mywedge \SN(M) \mywedge \NF(M)=\ctorapp{c}{\vec{v'}}{\vec{v}} \mywedge\\
%   \Sigma(c)=\ctortp{\Gammap}{\pitype{\vec{x}}{\vec{A}}{\ctorapp{a^i}{\Gammap}{\vec{N'}}}} \mywedge\\
%   \forall i. \pair{v_i}{I_i}\in \interp[\Delta,\Gammap\mapsto \interp{\vec{M}},x_{<i}\mapsto I_{<i};\Gamma,\Gammap,x_{<i}:A_{<i}]{A_i}
% }
\\
& \cup &
\setcompr{\triple{\Gamma}{M}{\dummy}}{
  \typejd{(\Gamma,M)}{\ctorapp{a^{i}}{\vec{M}}{\vec{N}}}
  \mywedge \SN(M) \mywedge 
  \not\exists c.\NF(M)=\ctorapp{c}{\vec{v'}}{\vec{v}}
}
\\[10pt]

% variables
\interp[\Delta,x:A\mapsto\pair{N}{I},\Delta']{x} & = & I\\
\interp[\Delta,x:A,\Delta']{x} & = & \dummy
\\

% applications
\interp{M\;N} & = & \appstar{\interp{M}}{\triple{\cdot}{N}{\interp{N}}}\\

% lambdas
\interp{\lamabs{x}{A}{M}} & = &
\lamstar{\triple{\Gamma}{N}{I}}{\interpd{A}}{\interp[\Delta,\Gamma,x:A\mapsto \pair{N}{I}]{M}}
\\

% constructors
\interp{\ctorapp{c^{-1}}{\vec{M}}{\vec{N}}} & = & \dummy\\
\interp{\ctorapp{c^i}{\vec{M}}{\vec{N}}} & = & \pair{c}{\interp{\vec{N}}}\\

% pm-functions
\interp{\pmfuni{-1}{x}{\Gammaarg}{\pc{\vec{P}}{\vec{\Gamma}}{\vec{M}}}} & = &
\dummy
\\

\interp{\pmfuni{i}{x}{\Gammaarg}{\pc{\vec{P}}{\vec{\Gamma}}{\vec{M}}}} & = &
\primrec{f}{
  \begin{array}[t]{@{}l@{}}
    \lamstar{\vec{p}}{\interp{\Gammaarg}}{
      \lamstar{\pair{M}{I}}{\interp[\Delta,\Gammaarg\mapsto\vec{p}]{\ctorapp{a}{\vec{N}}{\vec{Q}}}}{\\
        \ind{1} \mathsf{match}\;\pair{\NF(M)}{I}\;\mathsf{with}\\
        \ind{2}\ldots |\;
        \pair{\ctorapp{c_i}{\vec{N}}{\vec{R}}}{\pair{c_i}{\vec{I}}} \to \interp[\Delta,\Gammaarg\mapsto\vec{p},x:A_{\mathsf{fun}}\mapsto f,\Gamma_i\mapsto\pair{\vec{R}}{\vec{I}}]{M_i}
        \;| \ldots\\
        \ind{2}|\;\dummy \to \dummy
      }}
  \end{array}
}

\end{array}
\end{math}
\caption{Uniform Logical Relations Construction for CiC}
\label{fig:cic-lr}
\end{figure*}



The following lemmas all follow directly by induction:

\begin{lemma}[Weakening]
\label{lemma:interp-weakening}
$\interp[\Delta_1,\Delta_2,\Delta_3]{M}=\interp[\Delta_1,\Delta_3]{M}$.
\end{lemma}


\begin{lemma}[Substitution]
  \label{lemma:interp-subst}
  $\interp[\Delta,x:A\mapsto\pair{N}{\interp{N}},\Delta']{M}=
  \interp[\Delta,\Delta']{[N/x]M}$.
\end{lemma}


\begin{lemma}
  \label{lemma:delta-eq}
  If $M\conv M'$ then
  $\interp[\Delta,x\mapsto\pair{M}{I},\Delta']{N}=
  \interp[\Delta,x\mapsto\pair{M'}{I},\Delta']{N}$.
\end{lemma}

\begin{lemma}
  \label{lemma:interp-eq}
  If $M\conv M'$ then $\interp{M}=\interp{M'}$.
\end{lemma}


\begin{lemma}
  \label{lemma:sort-subtyping}
  If $s_1\subtype s_2$ then $\interp{s_1}\subseteq\interp{s_2}$.
\end{lemma}

\begin{lemma}
  \label{lemma:subtyping}
  If $A_1\subtype A_2$ then $\interp{A_1}\subseteq\interp{A_2}$.
\end{lemma}



Discussion: show that $\interp[\cdot]{\cdot}$ is well-defined
\begin{itemize}
\item by induction first on the inductive types in order, then on the
  sizes of the interpretations of the inductive types, then on the
  maximum sort in a term, then on the term being interpreted
\item interpretations of sorts decrease the maximum sort in a term;
  for $\Type_i$ we need that $\interp{M}=\interp{\NF(M)}$
  and values decrease the maximum sort in a term
\item interpretations of inductive types decrease the size of the
  inductive type and/or go backwards in the signature
\item pattern-matches do well-formed primitive recursion
\end{itemize}




We say that $\Delta$ is well-formed, written $\vdash\Delta$, iff the
context for $\Delta$ is well-formed, i.e.,
$\validj[\cdot]{\dctxt{\Delta}}$, and for all decompositions
$\Delta=\Delta_1,x:A\mapsto\pair{M}{I}$ we have
$\triple{\cdot}{M}{I}\in\interpd[\Delta_1]{A}$.


\begin{lemma}
  \label{lemma:sorts}
  If $\typejd{(\Gamma,s_1)}{s_2}$ then
  $\triple{\Gamma}{s_1}{\interp{s_1}}\in\interp{s_2}$ for all $\Gamma$
  with $\validjd{\Gamma}$.
\end{lemma}

\begin{proof}
  Straightforward, as $\SN(s_1)$ is immediate and
  $\CR{s_1}{\interp{s_1}}$ by Lemma \ref{lemma:cr-sn}.
\end{proof}


\begin{lemma}
  \label{lemma:pi}
  Let $\typejd{A}{s_A}$, $\typejd{B}{s_B}$, and
  $\typejd{(\pitype{x}{A}{B})}{s}$.  If
  $\triple{\Gamma}{A}{\interp{A}}\in\interpd{s_A}$ and for all
  $\triple{\Gamma'}{N}{I}\in\interpd{A}$ we have
  $\triple{\cdot}{B}{\interp[\Delta,\Gamma,Gamma',x:A\mapsto\pair{N}{I}]{B}}\in\interp[\Delta,x:A\mapsto\pair{N}{I}]{s_B}$,
  then
  $\triple{\cdot}{\pitype{x}{A}{B}}{\interp{\pitype{x}{A}{B}}}\in\interp{s}$.
\end{lemma}

\begin{proof}
  FIXME HERE
  The non-trivial part is showing that
  $\CR{\pitype{x}{A}{B}}{\interpd{\pitype{x}{A}{B}}}$:
  \begin{description}
  \item[\textbf{(CR 1):}] \hspace*{10pt} If
    $\triple{\Gamma}{M}{F}\in\interp{\pitype{x}{A}{B}}$ then we trivially have
    $\SN(M)$.
  \item[\textbf{(CR 2):}] \hspace*{10pt} If
    $\pair{M}{F}\in\interp{\pitype{x}{A}{B}}$ and $M\rrto M'$ then
    trivially $\SN(M')$. Also, for all $\pair{N}{I}\in\interp{A}$
    we have $\pair{M\;N}{\appstar{F}{\pair{N}{I}}}\in
    \interp[\Delta,x:A\mapsto\pair{N}{I}]{B}$ and so by \CRtwo\ for
    $B$ we have $\pair{M'\;N}{\appstar{F}{\pair{N}{I}}}\in
    \interp[\Delta,x:A\mapsto\pair{N}{I}]{B}$.
  \item[\textbf{(CR 3):}] \hspace*{10pt} Let $M$ be neutral and let
    $\pair{M'}{F}\in\interp{\pitype{x}{A}{B}}$ for all $M'$ such that
    $M\rrto M'$. Trivially we have that $\SN(M')$ for all $M'$ such
    that $M\rrto M'$, so $\SN(M)$. If $\interp{A}$ is empty, then this
    is all we need; otherwise let $\pair{N}{I}\in\interp{A}$. We must
    now show that $\pair{M\;N}{\appstar{F}{\pair{N}{I}}}\in
    \interp[\Delta,x:A\mapsto\pair{N}{I}]{B}$. By \CRone\ for
    $\interp{A}$, we have $\SN(N)$, so we can reason by induction on
    $\size{N}$. Now consider all terms reacable from $M\;N$ by one
    step of reduction. Since $M$ is neutral, this includes terms
    $M'\;N$ such that $M\rrto M'$, which are in
    $\interp[\Delta,x:A\mapsto\pair{N}{I}]{B}$ by induction, and terms
    $M\;N'$ such that $N\rrto N'$, which are in
    $\interp[\Delta,x:A\mapsto\pair{N}{I}]{B}$ by the induction
    hypothesis.  We then have the desired result by \CRthree\ for
    $\interp[\Delta,x:A\mapsto\pair{N}{I}]{B}$.
  \end{description}
\end{proof}



\begin{lemma}
  \label{lemma:ind-types}
  Let $\typejd{\vec{M},\vec{N}}{\Gammap,\Gammaarg}$ for
  $a:\pigamma{\Gammap}{\pigamma{\Gammaarg}{s}}$.  If
  $\pair{\vec{M},\vec{N}}{\interp{\vec{M},\vec{N}}}\in\interp{\Gammap,\Gammaarg}$
  then
  $\pair{\ctorapp{a}{\vec{M}}{\vec{N}}}{\interp{\ctorapp{a}{\vec{M}}{\vec{N}}}}\in
  \interp{s}$.
\end{lemma}

\begin{proof}
  Again, the main requirement is showing that
  $\CR{\ctorapp{a}{\vec{M}}{\vec{N}}}{\interpd{\ctorapp{a}{\vec{M}}{\vec{N}}}}$,
  which follows directly from Lemma \ref{lemma:cr-sn} for both the
  impredicative $a^{-1}$ and the predicative $a^i$.
\end{proof}


FIXME HERE

Remaining lemmas:
\begin{itemize}
\item Variables
\item Application
\item Abstraction
\item Constructors
\item Pattern-matching
\end{itemize}

\begin{theorem}[Reducibility]
  Let $\typej{M}{A}$ and $\typej{A}{s}$. If
  $\typej[\cdot]{\vec{N}}{\Gamma}$ and
  $\pair{\vec{N}}{\vec{I}}\in\interp[\cdot]{\Gamma}$ then
  $\interp[\Gamma\mapsto\pair{\vec{N}}{\vec{I}}]{A}\in\interp[\Gamma\mapsto\pair{\vec{N}}{\vec{I}}]{s}$ implies
  $\interp[\Gamma\mapsto\pair{\vec{N}}{\vec{I}}]{M}\in\interp[\Gamma\mapsto\pair{\vec{N}}{\vec{I}}]{A}$.
\end{theorem}

\begin{corollary}
If $\typej{M}{A}$ for any $\Gamma$ then $\SN(M)$.
\end{corollary}

\begin{proof}
FIXME
\end{proof}



%%%%%%%%%%%%%%%%%%%
%%%%% Section %%%%%
%%%%%%%%%%%%%%%%%%%
\section{The Calculus of Nominal Inductive Constructions (CNIC)}
\label{sec:cnic}



%%%%%%%%%%%%%%%%%%%
%%%%% Section %%%%%
%%%%%%%%%%%%%%%%%%%
\section{Strong Normalization for CNIC}
\label{sec:cnic-sn}



%%%%%%%%%%%%%%%%%%%
%%%%% Section %%%%%
%%%%%%%%%%%%%%%%%%%
\section{Conclusion}
\label{sec:conclusion}

The conclusion goes here.



% bibliography
\bibliographystyle{IEEEtran}
\bibliography{bib}

% that's all folks
\end{document}
